%-*- coding: UTF-8 -*-
% 文件名:LaTeX入门刘海洋.tex
% 描述:从 25week10 开始学习,因此资源存放在 25week10。
\documentclass[UTF8]{ctexart}

\usepackage{graphicx}

\title{文档标题}
\author{Lan}
\date{\today}

\bibliographystyle{plain}

\begin{document}

\maketitle
\tableofcontents


IATEX入门
電子工業出版社
…
PUILISHING HOUBE OF ELECTRONICS INDUSTRYhttp://www.phel.com.cn
IATEX
人门
上架建议:计算机/办公软件
ISBN 978-7-121-20208-7
9 787121 202087
定价:79.00元
Broadview	博文视点·IT出版旗舰品牌
www. broadview. com. cn	技术凝聚实力·专业创新出版
本书从介绍TEX到IATEX的发展开始,详细讲述了使用IATEX撰写科技文档以及幻灯片演示的基本方法和技巧。书的内容深入浅出,对于IATEX初学者以及科技工作者都是一本很好的参考书籍。
刘利刚,中国科学技术大学,教授
本书两大特色,“全”与“新”:不仅全面涵盖了LATEX的基础知识,而且包含了近几年IATEX的最新发展,内容丰富,层次分明,是一本很好的新手入门教程,对有一定使用经验的老手而言,也是一本完善的案头参考书。
韩建成, CTeX, 版主
很多IATEX用户对IATEX的认识依然很模糊,这本书会让IATEX入门变得清晰而专业。
虽然我只阅读了样稿,不得不说作者写得非常用心、认真。对IATEX入门来说,这是不可多得的好书、必读书。
王昭礼, ChinaTeX, 版主
策划编辑:张月萍
责任编辑:高洪霞
封面设计:张	昱
内	容	简	介
LaTeX 已经成为国际上数学、物理、计算机等科技领域专业排版的实际标准,其他领域(化学、生物、工程、语言学等)也有大量用户。本书内容取材广泛,涵盖了正文组织、自动化工具、数学公式、图表制作、幻灯片演示、错误处理等方面。考虑到LaTeX也是不断进化的,本书从数以千计的LaTeX工具宏包中进行甄选,选择较新而且实用的版本来讲解排版技巧。
为了方便读者的学习,本书给出了大量的实例和一定量的习题,并且还提供了案例代码。书中的示例大部分来自作者多年的实际排版案例,读者不断练习,肯定能掌握LaTeX的排版技能。
本书适合数学、物理、计算机、化学、生物、工程等专业的学生、工程师和教师阅读,也适合中学数学教师。此外,本书还适合对LaTeX排版有兴趣的人员。
未经许可,不得以任何方式复制或抄袭本书之部分或全部内容。
版权所有,侵权必究。
图书在版编目 (CIP)数据
LaTeX入门/刘海洋编著。—北京:电子工业出版社,2013.6
ISBN 978-7-121-20208-7
Ⅰ. ①L……Ⅱ. ①刘…Ⅲ。①排版一应用软件 Ⅳ.①TS803.23
中国版本图书馆CIP数据核字(2013) 第079359号
策划编辑:张月萍
责任编辑:高洪霞
印   刷:北京京科印刷有限公司
装   订:三河市皇庄路通装订厂
出版发行:电子工业出版社
北京市海淀区万寿路173信箱  邮编100036
开   本:787×980  1/16  印张:36.25  字数:632千字
印   次:2013年6月第1次印刷
印   数:4000册   定价:79.00元
凡所购买电子工业出版社图书有缺损问题,请向购买书店调换。若书店售缺,请与本社发行部联系,联系及邮购电话:(010)88254888.
质量投诉请发邮件至zlts@phei.com. cn                       dbqq@phei. com. cn.
服务热线:(010)88258888.
IATEX入门
刘海洋

\section{序}%
\label{sec:序}

看了本书的样稿后使人感到印象深刻。本书充分反映了TEX的最新进展,尽管TEX的生命力是顽强的,TEX的基本命令系统也是稳定的,但是它对非西方语言的扩展以及输出格式等都随着计算机技术的发展以及科技文献传播方式的变化而不断推陈出新,这也正是TEX能经久不衰的生命力所在。因此推广TEX的书也需要与时俱进。我们写的《KTEX入门与提高》的第二版至今已有7年了,可惜它的作者或者已退休,或者兴趣转移,不可能再作更新。我一直期待能有人出来写一本反映最新发展的TEX入门书作为我们那本书的补充及更新。现在看到了刘海洋的《IETEX入门》,觉得这正是我所期望的,甚至超过了我的期望。本书文笔活泼,阅读起来像是面对一位向你细细讲解的和蔼老师,他了解你的需求和会遇到的难点,使你爱不择手,而不像一般的软件说明书,只管板着脸罗列一大堆用法,不管你是否需要或是否理解。但是本书作者又很严谨,许多内容都有出处,好像一篇科研论文。不过说到底,这是一本面向读者需求的学习指导书,并非TEX的说明书。这正是想学习TEX的人最想要的书。而且第8章还讲到了更深入的技巧。因此本书的适用范围可以从初学者直至想自己设计版面或宏的高级应用者。大家都能从本书学到很多东西。尽管国内在TEX的普及与发展方面与西方发达国家相比还有很大的差距,但是感谢许多热心的TEX爱好者及他们的网站的努力,TEX在中国的推广也是富有成效的。越来越多的研究生用TEX写作论文或向期刊投稿,并且在答辩或演示时也广泛使用TEX生成的PDF。希望本书的出版能为TEX在中国的普及作出新的贡献。

陈志杰
华东师范大学数学系教授
2013年3月5日

iii
前言
提到ETEX,便不能不说起它的基础TEX.TEX是诞生于20世纪70年代末到80年代初的一款计算机排版软件,用来排版高质量的书籍,特别是包含有数学公式的书籍[124;126].TEX以追求高质量为目标,很早就实现了矢量描述的计算机字体、细致的分页断行算法和数学排版功能,因其数学排版能力得到了学术界的广泛使用,也启发了不少后来复杂的商业计算机排版软件。有趣的是,这样一款排版软件却并非在排版业界产生,而是由计算机科学家高德纳教授在修订其七卷本巨著《计算机程序设计艺术》的前三卷[[127-129]时,为了排版这一部书籍而产生的。这是一部花费高德纳几乎毕生精力的巨著,直到今天仍在撰写,然而在照相排版技术刚刚兴起的1976年,新的计算机系统却无法提供与传统手工排版相媲美的质量。面对这种情况,高德纳抱怨道[130]:
我不知道怎么办。我花了整整15年写这些书,可要是这么难看,我就再也不写了。我怎么能对这样的作品引以为豪呢?
从翌年开始,高德纳就在其学生、友人的帮助下,开发TEX排版软件。直到8年后TEX软件功能完备,他才又回到撰写书籍的工作中去。这段历史一直被引为TEX和高德纳的传奇,有“十年磨一剑”之称。TEX原本是用于个人的排版软件,这也引出了TEX与其他专业排版软件的一点重大的区别,就是TEX主要是由书籍、文章的作者本人来使用的,它是面向作者的。因此,TEX有许多方便作者的自定义功能,使用也简单方便,很快受到作者们的青睐,排版自己的学术书籍。
IATEX肇始于20世纪80年代初,也是Leslie Lamport博士为了编写他自己的一部书籍而设计的[137].ETEX实际上就是用TEX语言编写的一组宏代码,拥有比原来的TEX格式(PlainTEX)更为规范的命令和一整套预定义的格式,隐藏了不少排版方面的细节,可以让完全不懂排版理论的学者们也可以比较容易地将书籍和文稿排版出来。ETEX一出,很快更为风靡,在1994年KTEX28完善之后,现在已经成为国际上数学、物理、计算机等科技领域专业排版的事实标准,其他领域(化学、生物、工程、语言学等)也有大量用户。相关专业的学术期刊也都主要接受IATEX作为投稿格式。
既然TEX/IETEX主要是面向作者本人的排版软件,本书的目标对象也就以学术文章的作者为主,也就是需要经常编写LTEX稿件的高校师生和科研院所的研究人员。本书的内容选择以满足学术排版需求为准,阅读本书后读者应该不仅能应对各种学术投稿的简单需要,也将有能力排版一般的学术书籍,并使用ETEX完成简单的学术报告幻灯片。不过,本书也力图广泛取材,让排版公司的工人、中学数学教师或是用IETEX作笔

前言	v
记的电脑程序员都能有所得。
本书虽然名为“入门”,假定读者没有任何使用TEX的经验,但为了避免读者逡巡于门外而不入,也力图使内容详实可靠,为更深入地使用ETEX打好基础。在编写本书时,作者追求以下几个目标:
·内容广泛	本书从软件安装和最基本的示例讲起,然后按正文组织、自动化工具、数学公式、图表制做、幻灯片演示、错误处理等方面详述IATEX的功能和使用,最后收束于ETEX的扩展、相关工具和资源。ETEX的基本内容并不多,功能也很有限,但经过20多年的发展,现代IATEX文档的一大特点是大量使用工具宏包来完成复杂的工作。本书也力图体现这一特点,全书过半的篇幅都在讲解各种重要的ETEX宏包和工具。本书正文共有566页,作为一本入门书已是嫌多,但仍不可能包罗LTEX的所有方面,未免有遗珠之憾,只能留待读者自己学习了。
·取材从新TEX最初的一个设计目标是良好的稳定性,希望在多年前编写的文档在最新的系统中排版仍能得到完全相同的结果,各种排版命令的语义保持稳定,TEX也确实做到了这一点。然而ETEX是一个更为开放的系统,与其他软件一样,它是在不断进化的。不仅其内核从最初的ETEX 2.09到ETEX2E再到正在开发中的IATEX3不断变化,而且还有数以千计的工具宏包在不断更新,完成各种复杂的排版功能。实现TEX语言的TEX引擎,也在不断增添新的功能。为了反映这种变化,本书作者也尽量对内容加以甄别,选取较新并且实用的软件工具加以介绍。
·切合实用	为了增强实用性,本书给出了大量实例和一定量的习题。第1章和第6章提供了两段完整的文档源代码,而其他章节也给出了大量的代码示例。代码示例和习题很多都源自作者历年来收集的各类实际的排版问题,相信对于本书的读者也会有所裨益。
为了照顾不同层次的读者,本书按ETEX的不同功能编排章节,章节之间没有严格的顺序关系,阅读本书也不必完全依照章节顺序。
·希望快速上手的初学者应首先阅读第1章,安装好TEX软件并在1.2节学习基本的实例,然后就可以模仿实例编写自己的IATEX文档了,等到实际需要时再翻到对应的章节了解具体内容。
·希望系统学习IATEX的读者可以从前往后依次阅读。书中一些段落前,或整个一节之前有一个危险标记,说明这一段或一节内容较难或者依赖后面章节的内容,在初次阅读时可以略过,可以在读完基本内容后再来了解这部分内容。还有一些段落前有两个危险标记,则表示这些内容中部分已经超出本书的范围,通常需要参见书中引用的其他文档才能完全理解。
vi
·具有一定ETEX经验的读者可以根据自己的需要查找有用的内容,书后的索引将有助于找到特定的概念或命令,而每章末尾的注记与书后的文献列表则可以帮助读者找到本书中未能详述的内容。
本书使用不同的字体表示不同的内容。在正文中,使用等宽字体表示代码,如 $\alpha$ 命令、equation环境;用无衬线字体表示宏包名称,如amsmath宏包、beamer文档类;用尖括号内的楷体(西文斜体)表示参数,如(长度)、(key)。在表示ETEX命令或环境的语法形式时,则使用加粗的等宽字体,如:

(编码)〈族〉〈系列)(形状〉

书中给出了大量示例代码。大部分示例以左右对照的方式给出,左侧灰色框中是代码,右侧白色框中是代码的排版效果,例如:
\[
  \Delta = b^{2} - 4 a c0-1	 \Delta = b^2-4ac
\]
较长的示例则以上下对照的方式给出,如:
\[
x {1,2} = \frac{-b \pm \sqrt{b^2-4ac}}{2a}
0-2	\]
\[
  x_{1 , 2} = \frac{ - b \pm \sqrt{b^{2} - 4 a c}}{2 a}
\]
还有一些代码示例没有直接的排版结果,则只给出源代码。如上所示,示例通常会有一个编号以方便引用。本书中所有带编号的示例和第1章、第6章的两个大的例子会随书附带,也可以在CTEX论坛网站上获取。
书中在部分章节后面安排了一些题外的内容,在标题前用书籍符号标示(如右),内容用楷体字印刷。这些内容游离于本书的主线之外,主要介绍一些背景知识,读者可根据自己的兴趣选择阅读。
此外,在部分章节后还设置了少量的练习题,用铅笔符号标示(如右),读者可据此检查自己是否掌握了正文中的内容。这些题目并非为了把读者难住,大部分练习在书末都有解答或提示。
练	习
在本书编写过程中,许多朋友都为作者提供了无私的帮助。韩建成阅读了本书早期的草稿和初稿,在结构和内容方面都提出了宝贵的意见和建议;赵劲松和李清阅读了本书的初稿,并在内容上给出了详细的建议与勘误;江疆和王越在阅读初稿后,对
前言	vii
本书的内容和格式都提出了宝贵的意见。本书的编写一直受到在CTEX论坛与水木社区TEX版上网友们的关注和支持,论坛中对IATEX具体问题的大量讨论时常能启发作者的思路,为成书提供了重要的素材。在此,作者向所有关心本书的人们致以真诚的感谢!
作者已尽力使本书准确可靠,但受精力和水平所限,书中的错误在所难免。欢迎读者指出书中的技术上的、文字上的或是排版上的任何错误。有关本书的各种问题,可发送电子邮件至info@dozan.cn联系本书出版策划。
刘海洋

\section{目录}%
\label{sec:目录}

序…………………………………………………………………………………………………………………………………………………………iii
前言⋯⋯⋯⋯⋯⋯⋯⋯⋯⋯⋯⋯⋯⋯⋯⋯⋯⋯⋯⋯⋯⋯⋯⋯⋯⋯⋯⋯⋯⋯⋯⋯⋯⋯⋯⋯⋯⋯⋯⋯⋯⋯⋯⋯⋯⋯⋯⋯⋯⋯ iv
第1章	熟悉IATpX⋯⋯⋯⋯⋯⋯⋯⋯⋯⋯⋯⋯⋯⋯⋯⋯⋯⋯⋯⋯⋯⋯⋯⋯⋯⋯⋯⋯⋯⋯⋯⋯⋯⋯⋯⋯⋯⋯⋯⋯⋯⋯⋯ 1
1.1 让IATEX跑起来⋯⋯⋯⋯⋯⋯⋯⋯⋯⋯⋯⋯⋯⋯⋯⋯⋯⋯⋯⋯⋯⋯⋯⋯⋯⋯⋯⋯⋯⋯⋯⋯⋯⋯⋯⋯⋯⋯⋯⋯⋯⋯⋯⋯⋯⋯⋯⋯⋯⋯⋯⋯⋯⋯ 2
1.1.1 ETEX的发行版及其安装⋯⋯⋯⋯⋯⋯⋯⋯⋯⋯⋯⋯⋯⋯⋯⋯⋯⋯⋯⋯⋯⋯⋯⋯⋯⋯⋯⋯⋯⋯⋯⋯⋯⋯⋯⋯⋯⋯⋯⋯⋯⋯⋯⋯⋯⋯⋯⋯⋯ 2
CTEX套装/3·TEX Live/7
1.1.2编辑器与周边工具⋯⋯⋯⋯⋯⋯⋯⋯⋯⋯⋯⋯⋯⋯⋯⋯⋯⋯⋯⋯⋯⋯⋯⋯⋯⋯⋯⋯⋯⋯⋯⋯⋯⋯⋯⋯⋯⋯⋯⋯⋯⋯⋯⋯⋯⋯⋯⋯⋯⋯⋯⋯⋯ 13
编辑器举例——TeXworks/13·PDF阅读器/18·命令行工具/21
1.1.3 "Happy TEXing"与“特可爱排版”⋯⋯⋯⋯⋯⋯⋯⋯⋯⋯⋯⋯⋯⋯⋯⋯⋯⋯⋯⋯⋯⋯⋯⋯⋯⋯⋯⋯⋯⋯⋯⋯⋯⋯⋯27
1.2从一个例子说起⋯⋯⋯⋯⋯⋯⋯⋯⋯⋯⋯⋯⋯⋯⋯⋯⋯⋯⋯⋯⋯⋯⋯⋯⋯⋯⋯⋯⋯⋯⋯⋯⋯⋯⋯⋯⋯⋯⋯⋯⋯⋯⋯⋯⋯⋯⋯⋯⋯⋯⋯⋯ 32
1.2.1 确定目标⋯⋯⋯⋯⋯⋯⋯⋯⋯⋯⋯⋯⋯⋯⋯⋯⋯⋯⋯⋯⋯⋯⋯⋯⋯⋯⋯⋯⋯⋯⋯⋯⋯⋯⋯⋯⋯⋯⋯⋯⋯⋯⋯⋯⋯⋯⋯⋯⋯⋯⋯⋯⋯⋯⋯⋯⋯⋯⋯⋯⋯ 32
1.2.2从提纲开始⋯⋯⋯⋯⋯⋯⋯⋯⋯⋯⋯⋯⋯⋯⋯⋯⋯⋯⋯⋯⋯⋯⋯⋯⋯⋯⋯⋯⋯⋯⋯⋯⋯⋯⋯⋯⋯⋯⋯⋯⋯⋯⋯⋯⋯⋯⋯⋯⋯⋯⋯⋯⋯⋯⋯⋯⋯⋯⋯ 32
1.2.3 填写正文⋯⋯⋯⋯⋯⋯⋯⋯⋯⋯⋯⋯⋯⋯⋯⋯⋯⋯⋯⋯⋯⋯⋯⋯⋯⋯⋯⋯⋯⋯⋯⋯⋯⋯⋯⋯⋯⋯⋯⋯⋯⋯⋯⋯⋯⋯⋯⋯⋯⋯⋯⋯⋯⋯⋯⋯⋯⋯⋯⋯⋯ 35
1.2.4命令与环境⋯⋯⋯⋯⋯⋯⋯⋯⋯⋯⋯⋯⋯⋯⋯⋯⋯⋯⋯⋯⋯⋯⋯⋯⋯⋯⋯⋯⋯⋯⋯⋯⋯⋯⋯⋯⋯⋯⋯⋯⋯⋯⋯⋯⋯⋯⋯⋯⋯⋯⋯⋯⋯⋯⋯⋯⋯⋯⋯ 36
1.2.5 遭遇数学公式⋯⋯⋯⋯⋯⋯⋯⋯⋯⋯⋯⋯⋯⋯⋯⋯⋯⋯⋯⋯⋯⋯⋯⋯⋯⋯⋯⋯⋯⋯⋯⋯⋯⋯⋯⋯⋯⋯⋯⋯⋯⋯⋯⋯⋯⋯⋯⋯⋯⋯⋯⋯⋯⋯⋯⋯⋯ 38
1.2.6使用图表⋯⋯⋯⋯⋯⋯⋯⋯⋯⋯⋯⋯⋯⋯⋯⋯⋯⋯⋯⋯⋯⋯⋯⋯⋯⋯⋯⋯⋯⋯⋯⋯⋯⋯⋯⋯⋯⋯⋯⋯⋯⋯⋯⋯⋯⋯⋯⋯⋯⋯⋯⋯⋯⋯⋯⋯⋯⋯ 39
1.2.7 自动化工具⋯⋯⋯⋯⋯⋯⋯⋯⋯⋯⋯⋯⋯⋯⋯⋯⋯⋯⋯⋯⋯⋯⋯⋯⋯⋯⋯⋯⋯⋯⋯⋯⋯⋯⋯⋯⋯⋯⋯⋯⋯⋯⋯⋯⋯⋯⋯⋯⋯⋯⋯⋯⋯⋯⋯⋯⋯⋯⋯ 43
1.2.8设计文章的格式⋯⋯⋯⋯⋯⋯⋯⋯⋯⋯⋯⋯⋯⋯⋯⋯⋯⋯⋯⋯⋯⋯⋯⋯⋯⋯⋯⋯⋯⋯⋯⋯⋯⋯⋯⋯⋯⋯⋯⋯⋯⋯⋯⋯⋯⋯⋯⋯⋯⋯⋯⋯⋯⋯⋯ 46
本章注记⋯⋯⋯⋯⋯⋯⋯⋯⋯⋯⋯⋯⋯⋯⋯⋯⋯⋯⋯⋯⋯⋯⋯⋯⋯⋯⋯⋯⋯⋯⋯⋯⋯⋯⋯⋯⋯⋯⋯⋯⋯⋯⋯⋯⋯⋯⋯⋯⋯⋯⋯⋯⋯⋯⋯⋯⋯⋯⋯⋯⋯⋯⋯ 49
第2章	组织你的文本⋯⋯⋯⋯⋯⋯⋯⋯⋯⋯⋯⋯⋯⋯⋯⋯⋯⋯⋯⋯⋯⋯⋯⋯⋯⋯⋯⋯⋯⋯⋯⋯⋯⋯⋯⋯⋯⋯⋯50
2.1 文字与符号⋯⋯⋯⋯⋯⋯⋯⋯⋯⋯⋯⋯⋯⋯⋯⋯⋯⋯⋯⋯⋯⋯⋯⋯⋯⋯⋯⋯⋯⋯⋯⋯⋯⋯⋯⋯⋯⋯⋯⋯⋯⋯⋯⋯⋯⋯⋯⋯⋯⋯⋯⋯⋯⋯⋯⋯⋯ 50
2.1.1 字斟句酌⋯⋯⋯⋯⋯⋯⋯⋯⋯⋯⋯⋯⋯⋯⋯⋯⋯⋯⋯⋯⋯⋯⋯⋯⋯⋯⋯⋯⋯⋯⋯⋯⋯⋯⋯⋯⋯⋯⋯⋯⋯⋯⋯⋯⋯⋯⋯⋯⋯⋯⋯⋯⋯⋯⋯⋯⋯⋯⋯⋯ 50
从字母表到单词/50·  正确使用标点/54·  看不见的字符————空格与换
行/57
2.1.2特殊符号⋯⋯⋯⋯⋯⋯⋯⋯⋯⋯⋯⋯⋯⋯⋯⋯⋯⋯⋯⋯⋯⋯⋯⋯⋯⋯⋯⋯⋯⋯⋯⋯⋯⋯⋯⋯⋯⋯⋯⋯⋯⋯⋯⋯⋯⋯⋯⋯⋯⋯⋯⋯⋯⋯⋯⋯⋯⋯⋯⋯⋯ 60
viii
目录	ix
2.1.3 字体⋯⋯⋯⋯⋯⋯⋯⋯⋯⋯⋯⋯⋯⋯⋯⋯⋯⋯⋯⋯⋯⋯⋯⋯⋯⋯⋯⋯⋯⋯⋯⋯⋯⋯⋯⋯⋯⋯⋯⋯⋯⋯⋯⋯⋯⋯⋯⋯⋯⋯⋯⋯⋯⋯⋯⋯⋯⋯⋯⋯⋯⋯⋯ 62
字体的坐标/62·使用更多字体/67·  强调文字/78
2.1.4字号与行距⋯⋯⋯⋯⋯⋯⋯⋯⋯⋯⋯⋯⋯⋯⋯⋯⋯⋯⋯⋯⋯⋯⋯⋯⋯⋯⋯⋯⋯⋯⋯⋯⋯⋯⋯⋯⋯⋯⋯⋯⋯⋯⋯⋯⋯⋯⋯⋯⋯⋯⋯⋯⋯⋯⋯⋯⋯⋯ 81
2.1.5水平间距与盒子⋯⋯⋯⋯⋯⋯⋯⋯⋯⋯⋯⋯⋯⋯⋯⋯⋯⋯⋯⋯⋯⋯⋯⋯⋯⋯⋯⋯⋯⋯⋯⋯⋯⋯⋯⋯⋯⋯⋯⋯⋯⋯⋯⋯⋯⋯⋯⋯⋯⋯⋯⋯⋯⋯⋯⋯ 85
水平间距/85·盒子/88
2.2段落与文本环境⋯⋯⋯⋯⋯⋯⋯⋯⋯⋯⋯⋯⋯⋯⋯⋯⋯⋯⋯⋯⋯⋯⋯⋯⋯⋯⋯⋯⋯⋯⋯⋯⋯⋯⋯⋯⋯⋯⋯⋯⋯⋯⋯⋯⋯⋯⋯⋯⋯⋯⋯⋯⋯⋯⋯⋯⋯ 91
2.2.1 正文段落⋯⋯⋯⋯⋯⋯⋯⋯⋯⋯⋯⋯⋯⋯⋯⋯⋯⋯⋯⋯⋯⋯⋯⋯⋯⋯⋯⋯⋯⋯⋯⋯⋯⋯⋯⋯⋯⋯⋯⋯⋯⋯⋯⋯⋯⋯⋯⋯⋯⋯⋯⋯⋯⋯⋯⋯⋯⋯⋯⋯ 91
2.2.2文本环境⋯⋯⋯⋯⋯⋯⋯⋯⋯⋯⋯⋯⋯⋯⋯⋯⋯⋯⋯⋯⋯⋯⋯⋯⋯⋯⋯⋯⋯⋯⋯⋯⋯⋯⋯⋯⋯⋯⋯⋯⋯⋯⋯⋯⋯⋯⋯⋯⋯⋯⋯⋯⋯⋯⋯⋯⋯⋯⋯⋯ 96
2.2.3列表环境⋯⋯⋯⋯⋯⋯⋯⋯⋯⋯⋯⋯⋯⋯⋯⋯⋯⋯⋯⋯⋯⋯⋯⋯⋯⋯⋯⋯⋯⋯⋯⋯⋯⋯⋯⋯⋯⋯⋯⋯⋯⋯⋯⋯⋯⋯⋯⋯⋯⋯⋯⋯⋯⋯⋯⋯⋯⋯⋯⋯ 97
基本列表环境/97·  计数器与编号/99·  定制列表环境/102
2.2.4定理类环境⋯⋯⋯⋯⋯⋯⋯⋯⋯⋯⋯⋯⋯⋯⋯⋯⋯⋯⋯⋯⋯⋯⋯⋯⋯⋯⋯⋯⋯⋯⋯⋯⋯⋯⋯⋯⋯⋯⋯⋯⋯⋯⋯⋯⋯⋯⋯⋯⋯⋯⋯⋯⋯⋯⋯⋯⋯⋯106
2.2.5抄录和代码环境⋯⋯⋯⋯⋯⋯⋯⋯⋯⋯⋯⋯⋯⋯⋯⋯⋯⋯⋯⋯⋯⋯⋯⋯⋯⋯⋯⋯⋯⋯⋯⋯⋯⋯⋯⋯⋯⋯⋯⋯⋯⋯⋯⋯⋯⋯⋯⋯⋯⋯⋯⋯⋯⋯109
抄录命令与环境/109·程序代码与listings/111
2.2.6 tabbing环境⋯⋯⋯⋯⋯⋯⋯⋯⋯⋯⋯⋯⋯⋯⋯⋯⋯⋯⋯⋯⋯⋯⋯⋯⋯⋯⋯⋯⋯⋯⋯⋯⋯⋯⋯⋯⋯⋯⋯⋯⋯⋯⋯⋯⋯⋯⋯⋯⋯⋯⋯⋯⋯⋯⋯⋯116
2.2.7脚注与边注⋯⋯⋯⋯⋯⋯⋯⋯⋯⋯⋯⋯⋯⋯⋯⋯⋯⋯⋯⋯⋯⋯⋯⋯⋯⋯⋯⋯⋯⋯⋯⋯⋯⋯⋯⋯⋯⋯⋯⋯⋯⋯⋯⋯⋯⋯⋯⋯⋯⋯⋯⋯⋯⋯⋯⋯⋯⋯⋯⋯118
2.2.8垂直间距与垂直盒子⋯⋯⋯⋯⋯⋯⋯⋯⋯⋯⋯⋯⋯⋯⋯⋯⋯⋯⋯⋯⋯⋯⋯⋯⋯⋯⋯⋯⋯⋯⋯⋯⋯⋯⋯⋯⋯⋯⋯⋯⋯⋯⋯⋯⋯⋯⋯⋯⋯⋯⋯121
2.3文档的结构层次⋯⋯⋯⋯⋯⋯⋯⋯⋯⋯⋯⋯⋯⋯⋯⋯⋯⋯⋯⋯⋯⋯⋯⋯⋯⋯⋯⋯⋯⋯⋯⋯⋯⋯⋯⋯⋯⋯⋯⋯⋯⋯⋯⋯⋯⋯⋯⋯⋯⋯⋯⋯⋯⋯127
2.3.1标题和标题页⋯⋯⋯⋯⋯⋯⋯⋯⋯⋯⋯⋯⋯⋯⋯⋯⋯⋯⋯⋯⋯⋯⋯⋯⋯⋯⋯⋯⋯⋯⋯⋯⋯⋯⋯⋯⋯⋯⋯⋯⋯⋯⋯⋯⋯⋯⋯⋯⋯⋯⋯⋯⋯⋯⋯⋯127
2.3.2划分章节⋯⋯⋯⋯⋯⋯⋯⋯⋯⋯⋯⋯⋯⋯⋯⋯⋯⋯⋯⋯⋯⋯⋯⋯⋯⋯⋯⋯⋯⋯⋯⋯⋯⋯⋯⋯⋯⋯⋯⋯⋯⋯⋯⋯⋯⋯⋯⋯⋯⋯⋯⋯⋯⋯⋯⋯⋯⋯⋯⋯129
2.3.3多文件编译⋯⋯⋯⋯⋯⋯⋯⋯⋯⋯⋯⋯⋯⋯⋯⋯⋯⋯⋯⋯⋯⋯⋯⋯⋯⋯⋯⋯⋯⋯⋯⋯⋯⋯⋯⋯⋯⋯⋯⋯⋯⋯⋯⋯⋯⋯⋯⋯⋯⋯⋯⋯⋯⋯132
2.3.4定制章节格式⋯⋯⋯⋯⋯⋯⋯⋯⋯⋯⋯⋯⋯⋯⋯⋯⋯⋯⋯⋯⋯⋯⋯⋯⋯⋯⋯⋯⋯⋯⋯⋯⋯⋯⋯⋯⋯⋯⋯⋯⋯⋯⋯⋯⋯⋯⋯⋯⋯⋯⋯⋯135
2.4文档类与整体格式设计⋯⋯⋯⋯⋯⋯⋯⋯⋯⋯⋯⋯⋯⋯⋯⋯⋯⋯⋯⋯⋯⋯⋯⋯⋯⋯⋯⋯⋯⋯⋯⋯⋯⋯⋯⋯⋯⋯⋯⋯⋯⋯⋯⋯⋯⋯⋯138
2.4.1基本文档类和ctex文档类⋯⋯⋯⋯⋯⋯⋯⋯⋯⋯⋯⋯⋯⋯⋯⋯⋯⋯⋯⋯⋯⋯⋯⋯⋯⋯⋯⋯⋯⋯⋯⋯⋯⋯⋯⋯⋯⋯⋯⋯⋯⋯⋯⋯⋯⋯138
2.4.2页面尺寸与geometry⋯⋯⋯⋯⋯⋯⋯⋯⋯⋯⋯⋯⋯⋯⋯⋯⋯⋯⋯⋯⋯⋯⋯⋯⋯⋯⋯⋯⋯⋯⋯⋯⋯⋯⋯⋯⋯⋯⋯⋯⋯⋯⋯⋯⋯⋯⋯⋯⋯⋯ 142
2.4.3页面格式与fancyhdr⋯⋯⋯⋯⋯⋯⋯⋯⋯⋯⋯⋯⋯⋯⋯⋯⋯⋯⋯⋯⋯⋯⋯⋯⋯⋯⋯⋯⋯⋯⋯⋯⋯⋯⋯⋯⋯⋯⋯⋯⋯⋯⋯⋯⋯⋯⋯⋯⋯⋯ 145
2.4.4分栏控制与multicol⋯⋯⋯⋯⋯⋯⋯⋯⋯⋯⋯⋯⋯⋯⋯⋯⋯⋯⋯⋯⋯⋯⋯⋯⋯⋯⋯⋯⋯⋯⋯⋯⋯⋯⋯⋯⋯⋯⋯⋯⋯⋯⋯⋯⋯⋯⋯⋯⋯⋯⋯ 149
2.4.5定义命令与环境⋯⋯⋯⋯⋯⋯⋯⋯⋯⋯⋯⋯⋯⋯⋯⋯⋯⋯⋯⋯⋯⋯⋯⋯⋯⋯⋯⋯⋯⋯⋯⋯⋯⋯⋯⋯⋯⋯⋯⋯⋯⋯⋯⋯⋯⋯⋯⋯⋯⋯⋯⋯⋯⋯⋯⋯151
本章注记⋯⋯⋯⋯⋯⋯⋯⋯⋯⋯⋯⋯⋯⋯⋯⋯⋯⋯⋯⋯⋯⋯⋯⋯⋯⋯⋯⋯⋯⋯⋯⋯⋯⋯⋯⋯⋯⋯⋯⋯⋯⋯⋯⋯⋯⋯⋯⋯⋯⋯⋯⋯⋯⋯⋯⋯⋯⋯⋯⋯⋯⋯⋯155
x
第3章	自动化工具⋯⋯⋯⋯⋯⋯⋯⋯⋯⋯⋯⋯⋯⋯⋯⋯⋯⋯⋯⋯⋯⋯⋯⋯⋯⋯⋯⋯⋯⋯⋯⋯⋯⋯⋯⋯⋯⋯⋯⋯⋯⋯⋯⋯⋯157
3.1 目录…………………………………………………………………………………………………………………………………………………………………………………………157
3.1.1 目录和图表目录⋯⋯⋯⋯⋯⋯⋯⋯⋯⋯⋯⋯⋯⋯⋯⋯⋯⋯⋯⋯⋯⋯⋯⋯⋯⋯⋯⋯⋯⋯⋯⋯⋯⋯⋯⋯⋯⋯⋯⋯⋯⋯⋯⋯⋯⋯⋯⋯⋯⋯⋯⋯⋯⋯⋯157
3.1.2控制目录内容⋯⋯⋯⋯⋯⋯⋯⋯⋯⋯⋯⋯⋯⋯⋯⋯⋯⋯⋯⋯⋯⋯⋯⋯⋯⋯⋯⋯⋯⋯⋯⋯⋯⋯⋯⋯⋯⋯⋯⋯⋯⋯⋯⋯⋯⋯⋯⋯⋯⋯⋯⋯⋯⋯⋯158
3.1.3定制目录格式⋯⋯⋯⋯⋯⋯⋯⋯⋯⋯⋯⋯⋯⋯⋯⋯⋯⋯⋯⋯⋯⋯⋯⋯⋯⋯⋯⋯⋯⋯⋯⋯⋯⋯⋯⋯⋯⋯⋯⋯⋯⋯⋯⋯⋯⋯⋯⋯⋯⋯⋯⋯⋯⋯⋯161
3.2交叉引用⋯⋯⋯⋯⋯⋯⋯⋯⋯⋯⋯⋯⋯⋯⋯⋯⋯⋯⋯⋯⋯⋯⋯⋯⋯⋯⋯⋯⋯⋯⋯⋯⋯⋯⋯⋯⋯⋯⋯⋯⋯⋯⋯⋯⋯⋯⋯⋯⋯⋯⋯⋯⋯⋯⋯⋯⋯165
3.2.1标签与引用⋯⋯⋯⋯⋯⋯⋯⋯⋯⋯⋯⋯⋯⋯⋯⋯⋯⋯⋯⋯⋯⋯⋯⋯⋯⋯⋯⋯⋯⋯⋯⋯⋯⋯⋯⋯⋯⋯⋯⋯⋯⋯⋯⋯⋯⋯⋯⋯⋯⋯⋯⋯⋯⋯⋯⋯⋯⋯165
3.2.2更多交叉引用⋯⋯⋯⋯⋯⋯⋯⋯⋯⋯⋯⋯⋯⋯⋯⋯⋯⋯⋯⋯⋯⋯⋯⋯⋯⋯⋯⋯⋯⋯⋯⋯⋯⋯⋯⋯⋯⋯⋯⋯⋯⋯⋯⋯⋯⋯⋯⋯⋯⋯⋯⋯⋯⋯⋯⋯167
3.2.3电子文档与超链接⋯⋯⋯⋯⋯⋯⋯⋯⋯⋯⋯⋯⋯⋯⋯⋯⋯⋯⋯⋯⋯⋯⋯⋯⋯⋯⋯⋯⋯⋯⋯⋯⋯⋯⋯⋯⋯⋯⋯⋯⋯⋯⋯⋯⋯⋯⋯⋯⋯⋯⋯⋯169
3.3 BɪBTEX与文献数据库⋯⋯⋯⋯⋯⋯⋯⋯⋯⋯⋯⋯⋯⋯⋯⋯⋯⋯⋯⋯⋯⋯⋯⋯⋯⋯⋯⋯⋯⋯⋯⋯⋯⋯⋯⋯⋯⋯⋯⋯⋯⋯⋯⋯⋯⋯⋯174
3.3.1 BIBTEX基础⋯⋯⋯⋯⋯⋯⋯⋯⋯⋯⋯⋯⋯⋯⋯⋯⋯⋯⋯⋯⋯⋯⋯⋯⋯⋯⋯⋯⋯⋯⋯⋯⋯⋯⋯⋯⋯⋯⋯⋯⋯⋯⋯⋯⋯⋯⋯⋯⋯⋯⋯⋯⋯⋯⋯174
3.3.2 JabRef与文献数据库管理⋯⋯⋯⋯⋯⋯⋯⋯⋯⋯⋯⋯⋯⋯⋯⋯⋯⋯⋯⋯⋯⋯⋯⋯⋯⋯⋯⋯⋯⋯⋯⋯⋯⋯⋯⋯⋯⋯⋯⋯⋯⋯⋯⋯⋯⋯⋯183
3.3.3 用natbib定制文献格式⋯⋯⋯⋯⋯⋯⋯⋯⋯⋯⋯⋯⋯⋯⋯⋯⋯⋯⋯⋯⋯⋯⋯⋯⋯⋯⋯⋯⋯⋯⋯⋯⋯⋯⋯⋯⋯⋯⋯⋯⋯⋯⋯⋯⋯⋯⋯⋯⋯ 187
3.3.4更多的文献格式⋯⋯⋯⋯⋯⋯⋯⋯⋯⋯⋯⋯⋯⋯⋯⋯⋯⋯⋯⋯⋯⋯⋯⋯⋯⋯⋯⋯⋯⋯⋯⋯⋯⋯⋯⋯⋯⋯⋯⋯⋯⋯⋯⋯⋯⋯⋯⋯⋯⋯⋯⋯⋯⋯193
3.3.5 文献列表的底层命令⋯⋯⋯⋯⋯⋯⋯⋯⋯⋯⋯⋯⋯⋯⋯⋯⋯⋯⋯⋯⋯⋯⋯⋯⋯⋯⋯⋯⋯⋯⋯⋯⋯⋯⋯⋯⋯⋯⋯⋯⋯⋯⋯⋯⋯⋯⋯⋯⋯⋯196
3.4 Makeindex与索引⋯⋯⋯⋯⋯⋯⋯⋯⋯⋯⋯⋯⋯⋯⋯⋯⋯⋯⋯⋯⋯⋯⋯⋯⋯⋯⋯⋯⋯⋯⋯⋯⋯⋯⋯⋯⋯⋯⋯⋯⋯⋯⋯⋯⋯⋯⋯⋯⋯200
3.4.1 制作索引⋯⋯⋯⋯⋯⋯⋯⋯⋯⋯⋯⋯⋯⋯⋯⋯⋯⋯⋯⋯⋯⋯⋯⋯⋯⋯⋯⋯⋯⋯⋯⋯⋯⋯⋯⋯⋯⋯⋯⋯⋯⋯⋯⋯⋯⋯⋯⋯⋯⋯⋯⋯⋯⋯⋯⋯⋯⋯⋯⋯⋯200
3.4.2 定制索引格式⋯⋯⋯⋯⋯⋯⋯⋯⋯⋯⋯⋯⋯⋯⋯⋯⋯⋯⋯⋯⋯⋯⋯⋯⋯⋯⋯⋯⋯⋯⋯⋯⋯⋯⋯⋯⋯⋯⋯⋯⋯⋯⋯⋯⋯⋯⋯⋯⋯⋯⋯⋯⋯⋯⋯⋯⋯205
索引环境与格式/205·Makeindex与格式文件/207
3.4.3词汇表及其他⋯⋯⋯⋯⋯⋯⋯⋯⋯⋯⋯⋯⋯⋯⋯⋯⋯⋯⋯⋯⋯⋯⋯⋯⋯⋯⋯⋯⋯⋯⋯⋯⋯⋯⋯⋯⋯⋯⋯⋯⋯⋯⋯⋯⋯⋯⋯⋯⋯⋯⋯⋯⋯⋯⋯⋯213
手工生成词汇表/213·使用glossaries宏包/215
本章注记⋯⋯⋯⋯⋯⋯⋯⋯⋯⋯⋯⋯⋯⋯⋯⋯⋯⋯⋯⋯⋯⋯⋯⋯⋯⋯⋯⋯⋯⋯⋯⋯⋯⋯⋯⋯⋯⋯⋯⋯⋯⋯⋯⋯⋯⋯⋯⋯⋯⋯⋯⋯⋯⋯⋯⋯⋯⋯⋯⋯⋯⋯⋯⋯⋯⋯219
第4章	玩转数学公式………………………………………………………………………………………………………221
4.1数学模式概说⋯⋯⋯⋯⋯⋯⋯⋯⋯⋯⋯⋯⋯⋯⋯⋯⋯⋯⋯⋯⋯⋯⋯⋯⋯⋯⋯⋯⋯⋯⋯⋯⋯⋯⋯⋯⋯⋯⋯⋯⋯⋯⋯⋯⋯⋯⋯⋯⋯⋯⋯221
4.2数学结构……………………………………………………………………………………………………………………………………225
4.2.1上标与下标⋯⋯⋯⋯⋯⋯⋯⋯⋯⋯⋯⋯⋯⋯⋯⋯⋯⋯⋯⋯⋯⋯⋯⋯⋯⋯⋯⋯⋯⋯⋯⋯⋯⋯⋯⋯⋯⋯⋯⋯⋯⋯⋯⋯⋯⋯⋯⋯⋯⋯⋯⋯⋯⋯⋯⋯⋯⋯225
4.2.2上下画线与花括号⋯⋯⋯⋯⋯⋯⋯⋯⋯⋯⋯⋯⋯⋯⋯⋯⋯⋯⋯⋯⋯⋯⋯⋯⋯⋯⋯⋯⋯⋯⋯⋯⋯⋯⋯⋯⋯⋯⋯⋯⋯⋯⋯⋯⋯⋯⋯⋯⋯⋯⋯229
4.2.3分式⋯⋯⋯⋯⋯⋯⋯⋯⋯⋯⋯⋯⋯⋯⋯⋯⋯⋯⋯⋯⋯⋯⋯⋯⋯⋯⋯⋯⋯⋯⋯⋯⋯⋯⋯⋯⋯⋯⋯⋯⋯⋯⋯⋯⋯⋯⋯⋯⋯⋯⋯⋯⋯⋯⋯⋯⋯⋯⋯⋯⋯⋯230
4.2.4根式⋯⋯⋯⋯⋯⋯⋯⋯⋯⋯⋯⋯⋯⋯⋯⋯⋯⋯⋯⋯⋯⋯⋯⋯⋯⋯⋯⋯⋯⋯⋯⋯⋯⋯⋯⋯⋯⋯⋯⋯⋯⋯⋯⋯⋯⋯⋯⋯⋯⋯⋯⋯⋯⋯⋯⋯233
目录	xi
4.2.5 矩阵⋯⋯⋯⋯⋯⋯⋯⋯⋯⋯⋯⋯⋯⋯⋯⋯⋯⋯⋯⋯⋯⋯⋯⋯⋯⋯⋯⋯⋯⋯⋯⋯⋯⋯⋯⋯⋯⋯⋯⋯⋯⋯⋯⋯⋯⋯⋯⋯⋯⋯⋯⋯⋯⋯⋯⋯⋯⋯⋯⋯⋯⋯⋯234
4.3符号与类型⋯⋯⋯⋯⋯⋯⋯⋯⋯⋯⋯⋯⋯⋯⋯⋯⋯⋯⋯⋯⋯⋯⋯⋯⋯⋯⋯⋯⋯⋯⋯⋯⋯⋯⋯⋯⋯⋯⋯⋯⋯⋯⋯⋯⋯⋯⋯⋯⋯⋯⋯⋯⋯⋯⋯237
4.3.1 字母表与普通符号⋯⋯⋯⋯⋯⋯⋯⋯⋯⋯⋯⋯⋯⋯⋯⋯⋯⋯⋯⋯⋯⋯⋯⋯⋯⋯⋯⋯⋯⋯⋯⋯⋯⋯⋯⋯⋯⋯⋯⋯⋯⋯⋯⋯⋯⋯⋯⋯⋯⋯⋯⋯237
4.3.2数学算子⋯⋯⋯⋯⋯⋯⋯⋯⋯⋯⋯⋯⋯⋯⋯⋯⋯⋯⋯⋯⋯⋯⋯⋯⋯⋯⋯⋯⋯⋯⋯⋯⋯⋯⋯⋯⋯⋯⋯⋯⋯⋯⋯⋯⋯⋯⋯⋯⋯⋯⋯⋯⋯⋯⋯⋯⋯⋯⋯⋯244
4.3.3二元运算符与关系符⋯⋯⋯⋯⋯⋯⋯⋯⋯⋯⋯⋯⋯⋯⋯⋯⋯⋯⋯⋯⋯⋯⋯⋯⋯⋯⋯⋯⋯⋯⋯⋯⋯⋯⋯⋯⋯⋯⋯⋯⋯⋯⋯⋯⋯⋯⋯⋯⋯⋯249
4.3.4括号与定界符⋯⋯⋯⋯⋯⋯⋯⋯⋯⋯⋯⋯⋯⋯⋯⋯⋯⋯⋯⋯⋯⋯⋯⋯⋯⋯⋯⋯⋯⋯⋯⋯⋯⋯⋯⋯⋯⋯⋯⋯⋯⋯⋯⋯⋯⋯⋯⋯⋯⋯⋯⋯⋯⋯⋯⋯255
4.3.5标点⋯⋯⋯⋯⋯⋯⋯⋯⋯⋯⋯⋯⋯⋯⋯⋯⋯⋯⋯⋯⋯⋯⋯⋯⋯⋯⋯⋯⋯⋯⋯⋯⋯⋯⋯⋯⋯⋯⋯⋯⋯⋯⋯⋯⋯⋯⋯⋯⋯⋯⋯⋯⋯⋯⋯⋯⋯⋯⋯⋯⋯⋯⋯258
4.4多行公式⋯⋯⋯⋯⋯⋯⋯⋯⋯⋯⋯⋯⋯⋯⋯⋯⋯⋯⋯⋯⋯⋯⋯⋯⋯⋯⋯⋯⋯⋯⋯⋯⋯⋯⋯⋯⋯⋯⋯⋯⋯⋯⋯⋯⋯⋯⋯⋯⋯⋯⋯⋯⋯⋯⋯⋯⋯262
4.4.1罗列多个公式⋯⋯⋯⋯⋯⋯⋯⋯⋯⋯⋯⋯⋯⋯⋯⋯⋯⋯⋯⋯⋯⋯⋯⋯⋯⋯⋯⋯⋯⋯⋯⋯⋯⋯⋯⋯⋯⋯⋯⋯⋯⋯⋯⋯⋯⋯⋯⋯⋯⋯⋯⋯⋯⋯⋯⋯⋯263
4.4.2拆分单个公式⋯⋯⋯⋯⋯⋯⋯⋯⋯⋯⋯⋯⋯⋯⋯⋯⋯⋯⋯⋯⋯⋯⋯⋯⋯⋯⋯⋯⋯⋯⋯⋯⋯⋯⋯⋯⋯⋯⋯⋯⋯⋯⋯⋯⋯⋯⋯⋯⋯⋯⋯⋯⋯⋯⋯⋯267
4.4.3将公式组合成块⋯⋯⋯⋯⋯⋯⋯⋯⋯⋯⋯⋯⋯⋯⋯⋯⋯⋯⋯⋯⋯⋯⋯⋯⋯⋯⋯⋯⋯⋯⋯⋯⋯⋯⋯⋯⋯⋯⋯⋯⋯⋯⋯⋯⋯⋯⋯⋯269
4.5精调与杂项⋯⋯⋯⋯⋯⋯⋯⋯⋯⋯⋯⋯⋯⋯⋯⋯⋯⋯⋯⋯⋯⋯⋯⋯⋯⋯⋯⋯⋯⋯⋯⋯⋯⋯⋯⋯⋯⋯⋯⋯⋯⋯⋯⋯⋯⋯⋯⋯⋯⋯⋯⋯⋯⋯⋯⋯⋯273
4.5.1公式编号控制⋯⋯⋯⋯⋯⋯⋯⋯⋯⋯⋯⋯⋯⋯⋯⋯⋯⋯⋯⋯⋯⋯⋯⋯⋯⋯⋯⋯⋯⋯⋯⋯⋯⋯⋯⋯⋯⋯⋯⋯⋯⋯⋯⋯⋯⋯⋯⋯⋯⋯⋯⋯⋯⋯⋯⋯273
4.5.2公式的字号⋯⋯⋯⋯⋯⋯⋯⋯⋯⋯⋯⋯⋯⋯⋯⋯⋯⋯⋯⋯⋯⋯⋯⋯⋯⋯⋯⋯⋯⋯⋯⋯⋯⋯⋯⋯⋯⋯⋯⋯⋯⋯⋯⋯⋯⋯⋯⋯⋯⋯⋯⋯⋯⋯⋯⋯⋯⋯276
4.5.3断行与数学间距⋯⋯⋯⋯⋯⋯⋯⋯⋯⋯⋯⋯⋯⋯⋯⋯⋯⋯⋯⋯⋯⋯⋯⋯⋯⋯⋯⋯⋯⋯⋯⋯⋯⋯⋯⋯⋯⋯⋯⋯⋯⋯⋯⋯⋯⋯⋯⋯⋯⋯278
本章注记⋯⋯⋯⋯⋯⋯⋯⋯⋯⋯⋯⋯⋯⋯⋯⋯⋯⋯⋯⋯⋯⋯⋯⋯⋯⋯⋯⋯⋯⋯⋯⋯⋯⋯⋯⋯⋯⋯⋯⋯⋯⋯⋯⋯⋯⋯⋯⋯⋯⋯⋯⋯⋯⋯⋯⋯⋯⋯⋯⋯284
第5章	绘制图表⋯⋯⋯⋯⋯⋯⋯⋯⋯⋯⋯⋯⋯⋯⋯⋯⋯⋯⋯⋯⋯⋯⋯⋯⋯⋯⋯⋯⋯⋯⋯⋯⋯⋯⋯⋯⋯⋯⋯⋯⋯⋯285
5.1 ETEX中的表格⋯⋯⋯⋯⋯⋯⋯⋯⋯⋯⋯⋯⋯⋯⋯⋯⋯⋯⋯⋯⋯⋯⋯⋯⋯⋯⋯⋯⋯⋯⋯⋯⋯⋯⋯⋯⋯⋯⋯⋯⋯⋯⋯⋯⋯⋯⋯⋯⋯⋯⋯285
5.1.1 tabular和array⋯⋯⋯⋯⋯⋯⋯⋯⋯⋯⋯⋯⋯⋯⋯⋯⋯⋯⋯⋯⋯⋯⋯⋯⋯⋯⋯⋯⋯⋯⋯⋯⋯⋯⋯⋯⋯⋯⋯⋯⋯⋯⋯⋯⋯⋯⋯⋯⋯⋯⋯⋯ 285
5.1.2表格单元的合并与分割⋯⋯⋯⋯⋯⋯⋯⋯⋯⋯⋯⋯⋯⋯⋯⋯⋯⋯⋯⋯⋯⋯⋯⋯⋯⋯⋯⋯⋯⋯⋯⋯⋯⋯⋯⋯⋯⋯⋯⋯⋯⋯⋯⋯⋯⋯⋯⋯292
5.1.3定宽表格与tabularx⋯⋯⋯⋯⋯⋯⋯⋯⋯⋯⋯⋯⋯⋯⋯⋯⋯⋯⋯⋯⋯⋯⋯⋯⋯⋯⋯⋯⋯⋯⋯⋯⋯⋯⋯⋯⋯⋯⋯⋯⋯⋯⋯⋯⋯⋯⋯⋯⋯⋯ 298
5.1.4长表格与longtable⋯⋯⋯⋯⋯⋯⋯⋯⋯⋯⋯⋯⋯⋯⋯⋯⋯⋯⋯⋯⋯⋯⋯⋯⋯⋯⋯⋯⋯⋯⋯⋯⋯⋯⋯⋯⋯⋯⋯⋯⋯⋯⋯⋯⋯⋯⋯⋯⋯⋯⋯⋯ 300
5.1.5三线表与表线控制⋯⋯⋯⋯⋯⋯⋯⋯⋯⋯⋯⋯⋯⋯⋯⋯⋯⋯⋯⋯⋯⋯⋯⋯⋯⋯⋯⋯⋯⋯⋯⋯⋯⋯⋯⋯⋯⋯⋯⋯⋯⋯⋯⋯⋯⋯⋯⋯⋯⋯⋯⋯307
5.1.6 array宏包与列格式控制⋯⋯⋯⋯⋯⋯⋯⋯⋯⋯⋯⋯⋯⋯⋯⋯⋯⋯⋯⋯⋯⋯⋯⋯⋯⋯⋯⋯⋯⋯⋯⋯⋯⋯⋯⋯⋯⋯⋯⋯⋯⋯⋯⋯⋯⋯⋯⋯⋯314
5.1.7定界符与子矩阵⋯⋯⋯⋯⋯⋯⋯⋯⋯⋯⋯⋯⋯⋯⋯⋯⋯⋯⋯⋯⋯⋯⋯⋯⋯⋯⋯⋯⋯⋯⋯⋯⋯⋯⋯⋯⋯⋯⋯⋯⋯⋯⋯⋯⋯⋯⋯⋯⋯⋯⋯⋯⋯⋯⋯317
5.2插图与变换⋯⋯⋯⋯⋯⋯⋯⋯⋯⋯⋯⋯⋯⋯⋯⋯⋯⋯⋯⋯⋯⋯⋯⋯⋯⋯⋯⋯⋯⋯⋯⋯⋯⋯⋯⋯⋯⋯⋯⋯⋯⋯⋯⋯⋯⋯⋯⋯⋯⋯⋯⋯⋯⋯⋯321
5.2.1 graphicx与插图⋯⋯⋯⋯⋯⋯⋯⋯⋯⋯⋯⋯⋯⋯⋯⋯⋯⋯⋯⋯⋯⋯⋯⋯⋯⋯⋯⋯⋯⋯⋯⋯⋯⋯⋯⋯⋯⋯⋯⋯⋯⋯⋯⋯⋯⋯⋯⋯⋯⋯⋯⋯⋯⋯⋯322
5.2.2几何变换⋯⋯⋯⋯⋯⋯⋯⋯⋯⋯⋯⋯⋯⋯⋯⋯⋯⋯⋯⋯⋯⋯⋯⋯⋯⋯⋯⋯⋯⋯⋯⋯⋯⋯⋯⋯⋯⋯⋯⋯⋯⋯⋯⋯⋯⋯⋯⋯⋯⋯⋯⋯⋯⋯⋯⋯⋯⋯⋯⋯⋯331
5.2.3页面旋转⋯⋯⋯⋯⋯⋯⋯⋯⋯⋯⋯⋯⋯⋯⋯⋯⋯⋯⋯⋯⋯⋯⋯⋯⋯⋯⋯⋯⋯⋯⋯⋯⋯⋯⋯⋯⋯⋯⋯⋯⋯⋯⋯⋯⋯⋯⋯⋯⋯⋯⋯⋯⋯333
xii
5.3浮动体与标题控制⋯⋯⋯⋯⋯⋯⋯⋯⋯⋯⋯⋯⋯⋯⋯⋯⋯⋯⋯⋯⋯⋯⋯⋯⋯⋯⋯⋯⋯⋯⋯⋯⋯⋯⋯⋯⋯⋯⋯⋯⋯⋯⋯⋯⋯⋯⋯⋯⋯⋯335
5.3.1 浮动体⋯⋯⋯⋯⋯⋯⋯⋯⋯⋯⋯⋯⋯⋯⋯⋯⋯⋯⋯⋯⋯⋯⋯⋯⋯⋯⋯⋯⋯⋯⋯⋯⋯⋯⋯⋯⋯⋯⋯⋯⋯⋯⋯⋯⋯⋯⋯⋯⋯⋯⋯⋯⋯⋯⋯⋯⋯⋯⋯⋯⋯⋯335
5.3.2标题控制与caption宏包⋯⋯⋯⋯⋯⋯⋯⋯⋯⋯⋯⋯⋯⋯⋯⋯⋯⋯⋯⋯⋯⋯⋯⋯⋯⋯⋯⋯⋯⋯⋯⋯⋯⋯⋯⋯⋯⋯⋯⋯⋯⋯⋯⋯⋯⋯⋯⋯341
5.3.3并排与子图表⋯⋯⋯⋯⋯⋯⋯⋯⋯⋯⋯⋯⋯⋯⋯⋯⋯⋯⋯⋯⋯⋯⋯⋯⋯⋯⋯⋯⋯⋯⋯⋯⋯⋯⋯⋯⋯⋯⋯⋯⋯⋯⋯⋯⋯⋯⋯⋯⋯⋯⋯⋯⋯⋯⋯⋯⋯⋯351
5.3.4浮动控制与float宏包⋯⋯⋯⋯⋯⋯⋯⋯⋯⋯⋯⋯⋯⋯⋯⋯⋯⋯⋯⋯⋯⋯⋯⋯⋯⋯⋯⋯⋯⋯⋯⋯⋯⋯⋯⋯⋯⋯⋯⋯⋯⋯⋯⋯⋯⋯⋯⋯⋯⋯357
5.3.5文字绕排⋯⋯⋯⋯⋯⋯⋯⋯⋯⋯⋯⋯⋯⋯⋯⋯⋯⋯⋯⋯⋯⋯⋯⋯⋯⋯⋯⋯⋯⋯⋯⋯⋯⋯⋯⋯⋯⋯⋯⋯⋯⋯⋯⋯⋯⋯⋯⋯⋯⋯⋯⋯⋯⋯⋯⋯⋯⋯⋯⋯361
5.4使用彩色⋯⋯⋯⋯⋯⋯⋯⋯⋯⋯⋯⋯⋯⋯⋯⋯⋯⋯⋯⋯⋯⋯⋯⋯⋯⋯⋯⋯⋯⋯⋯⋯⋯⋯⋯⋯⋯⋯⋯⋯⋯⋯⋯⋯⋯⋯⋯⋯⋯⋯⋯⋯⋯⋯⋯⋯365
5.4.1彩色表格⋯⋯⋯⋯⋯⋯⋯⋯⋯⋯⋯⋯⋯⋯⋯⋯⋯⋯⋯⋯⋯⋯⋯⋯⋯⋯⋯⋯⋯⋯⋯⋯⋯⋯⋯⋯⋯⋯⋯⋯⋯⋯⋯⋯⋯⋯⋯⋯⋯⋯⋯⋯⋯⋯⋯369
5.5 绘图语言⋯⋯⋯⋯⋯⋯⋯⋯⋯⋯⋯⋯⋯⋯⋯⋯⋯⋯⋯⋯⋯⋯⋯⋯⋯⋯⋯⋯⋯⋯⋯⋯⋯⋯⋯⋯⋯⋯⋯⋯⋯⋯⋯⋯⋯⋯⋯⋯⋯⋯⋯⋯⋯⋯⋯⋯⋯⋯⋯⋯⋯⋯⋯⋯373
5.5.1 XY-pic与交换图表⋯⋯⋯⋯⋯⋯⋯⋯⋯⋯⋯⋯⋯⋯⋯⋯⋯⋯⋯⋯⋯⋯⋯⋯⋯⋯⋯⋯⋯⋯⋯⋯⋯⋯⋯⋯⋯⋯⋯⋯⋯⋯⋯⋯⋯⋯⋯⋯⋯⋯⋯⋯373
5.5.2 PSTricks与TikZ简介⋯⋯⋯⋯⋯⋯⋯⋯⋯⋯⋯⋯⋯⋯⋯⋯⋯⋯⋯⋯⋯⋯⋯⋯⋯⋯⋯⋯⋯⋯⋯⋯⋯⋯⋯⋯⋯⋯⋯⋯⋯⋯⋯⋯⋯⋯⋯⋯⋯379
PSTricks/380·pgf与TikZ/388
5.5.3 METAPOST与Asymptote简介⋯⋯⋯⋯⋯⋯⋯⋯⋯⋯⋯⋯⋯⋯⋯⋯⋯⋯⋯⋯⋯⋯⋯⋯⋯⋯⋯⋯⋯⋯⋯⋯⋯⋯⋯⋯⋯⋯⋯⋯⋯398
METAPOST/398   ·Asymptote/405
本章注记⋯⋯⋯⋯⋯⋯⋯⋯⋯⋯⋯⋯⋯⋯⋯⋯⋯⋯⋯⋯⋯⋯⋯⋯⋯⋯⋯⋯⋯⋯⋯⋯⋯⋯⋯⋯⋯⋯⋯⋯⋯⋯⋯⋯⋯⋯⋯⋯⋯⋯⋯⋯⋯⋯⋯⋯⋯⋯⋯⋯⋯⋯⋯⋯409
第6章	幻灯片演示⋯⋯⋯⋯⋯⋯⋯⋯⋯⋯⋯⋯⋯⋯⋯⋯⋯⋯⋯⋯⋯⋯⋯⋯⋯⋯⋯⋯⋯⋯⋯⋯⋯⋯⋯⋯⋯⋯⋯⋯⋯412
6.1 组织幻灯内容⋯⋯⋯⋯⋯⋯⋯⋯⋯⋯⋯⋯⋯⋯⋯⋯⋯⋯⋯⋯⋯⋯⋯⋯⋯⋯⋯⋯⋯⋯⋯⋯⋯⋯⋯⋯⋯⋯⋯⋯⋯⋯⋯⋯⋯⋯⋯⋯⋯⋯⋯⋯416
6.1.1 帧⋯⋯⋯⋯⋯⋯⋯⋯⋯⋯⋯⋯⋯⋯⋯⋯⋯⋯⋯⋯⋯⋯⋯⋯⋯⋯⋯⋯⋯⋯⋯⋯⋯⋯⋯⋯⋯⋯⋯⋯⋯⋯⋯⋯⋯⋯⋯⋯⋯⋯⋯⋯⋯⋯⋯⋯⋯⋯⋯⋯⋯⋯⋯⋯⋯⋯417
6.1.2标题与文档信息⋯⋯⋯⋯⋯⋯⋯⋯⋯⋯⋯⋯⋯⋯⋯⋯⋯⋯⋯⋯⋯⋯⋯⋯⋯⋯⋯⋯⋯⋯⋯⋯⋯⋯⋯⋯⋯⋯⋯⋯⋯⋯⋯⋯⋯⋯⋯⋯⋯⋯⋯⋯⋯⋯419
6.1.3分节与目录⋯⋯⋯⋯⋯⋯⋯⋯⋯⋯⋯⋯⋯⋯⋯⋯⋯⋯⋯⋯⋯⋯⋯⋯⋯⋯⋯⋯⋯⋯⋯⋯⋯⋯⋯⋯⋯⋯⋯⋯⋯⋯⋯⋯⋯⋯⋯⋯⋯⋯⋯⋯⋯⋯⋯⋯⋯⋯⋯420
6.1.4文献⋯⋯⋯⋯⋯⋯⋯⋯⋯⋯⋯⋯⋯⋯⋯⋯⋯⋯⋯⋯⋯⋯⋯⋯⋯⋯⋯⋯⋯⋯⋯⋯⋯⋯⋯⋯⋯⋯⋯⋯⋯⋯⋯⋯⋯⋯⋯⋯⋯⋯⋯⋯⋯⋯⋯⋯⋯⋯⋯⋯⋯⋯⋯423
6.1.5定理与区块⋯⋯⋯⋯⋯⋯⋯⋯⋯⋯⋯⋯⋯⋯⋯⋯⋯⋯⋯⋯⋯⋯⋯⋯⋯⋯⋯⋯⋯⋯⋯⋯⋯⋯⋯⋯⋯⋯⋯⋯⋯⋯⋯⋯⋯⋯⋯⋯⋯⋯⋯⋯⋯⋯⋯⋯⋯⋯424
6.1.6 图表⋯⋯⋯⋯⋯⋯⋯⋯⋯⋯⋯⋯⋯⋯⋯⋯⋯⋯⋯⋯⋯⋯⋯⋯⋯⋯⋯⋯⋯⋯⋯⋯⋯⋯⋯⋯⋯⋯⋯⋯⋯⋯⋯⋯⋯⋯⋯⋯⋯⋯⋯⋯⋯⋯⋯⋯⋯⋯⋯⋯⋯⋯⋯⋯425
6.2风格的要素⋯⋯⋯⋯⋯⋯⋯⋯⋯⋯⋯⋯⋯⋯⋯⋯⋯⋯⋯⋯⋯⋯⋯⋯⋯⋯⋯⋯⋯⋯⋯⋯⋯⋯⋯⋯⋯⋯⋯⋯⋯⋯⋯⋯⋯⋯⋯⋯⋯⋯⋯⋯⋯⋯427
6.2.1使用主题⋯⋯⋯⋯⋯⋯⋯⋯⋯⋯⋯⋯⋯⋯⋯⋯⋯⋯⋯⋯⋯⋯⋯⋯⋯⋯⋯⋯⋯⋯⋯⋯⋯⋯⋯⋯⋯⋯⋯⋯⋯⋯⋯⋯⋯⋯⋯⋯⋯⋯⋯⋯⋯⋯⋯⋯⋯⋯⋯427
6.2.2 自定义格式⋯⋯⋯⋯⋯⋯⋯⋯⋯⋯⋯⋯⋯⋯⋯⋯⋯⋯⋯⋯⋯⋯⋯⋯⋯⋯⋯⋯⋯⋯⋯⋯⋯⋯⋯⋯⋯⋯⋯⋯⋯⋯⋯⋯⋯⋯⋯⋯⋯⋯⋯⋯⋯⋯⋯⋯⋯⋯⋯⋯428
6.3动态展示……………………………………………………………………………………………………………………………………………………432
6.3.1覆盖浅说⋯⋯⋯⋯⋯⋯⋯⋯⋯⋯⋯⋯⋯⋯⋯⋯⋯⋯⋯⋯⋯⋯⋯⋯⋯⋯⋯⋯⋯⋯⋯⋯⋯⋯⋯⋯⋯⋯⋯⋯⋯⋯⋯⋯⋯⋯⋯⋯⋯⋯⋯⋯⋯⋯⋯⋯⋯432
6.3.2活动对象与多媒体⋯⋯⋯⋯⋯⋯⋯⋯⋯⋯⋯⋯⋯⋯⋯⋯⋯⋯⋯⋯⋯⋯⋯⋯⋯⋯⋯⋯⋯⋯⋯⋯⋯⋯⋯⋯⋯⋯⋯⋯⋯⋯⋯⋯⋯⋯⋯⋯⋯⋯⋯⋯435
目录	xiii
本章注记⋯⋯⋯⋯⋯⋯⋯⋯⋯⋯⋯⋯⋯⋯⋯⋯⋯⋯⋯⋯⋯⋯⋯⋯⋯⋯⋯⋯⋯⋯⋯⋯⋯⋯⋯⋯⋯⋯⋯⋯⋯⋯⋯⋯⋯⋯⋯⋯⋯⋯⋯⋯⋯⋯⋯⋯⋯⋯⋯438
第7章	从错误中救赎⋯⋯⋯⋯⋯⋯⋯⋯⋯⋯⋯⋯⋯⋯⋯⋯⋯⋯⋯⋯⋯⋯⋯⋯⋯⋯⋯⋯⋯⋯⋯⋯⋯⋯⋯⋯⋯⋯⋯440
7.1 理解错误信息⋯⋯⋯⋯⋯⋯⋯⋯⋯⋯⋯⋯⋯⋯⋯⋯⋯⋯⋯⋯⋯⋯⋯⋯⋯⋯⋯⋯⋯⋯⋯⋯⋯⋯⋯⋯⋯⋯⋯⋯⋯⋯⋯⋯⋯⋯⋯⋯⋯⋯⋯⋯441
7.1.1 与TEX交互⋯⋯⋯⋯⋯⋯⋯⋯⋯⋯⋯⋯⋯⋯⋯⋯⋯⋯⋯⋯⋯⋯⋯⋯⋯⋯⋯⋯⋯⋯⋯⋯⋯⋯⋯⋯⋯⋯⋯⋯⋯⋯⋯⋯⋯⋯⋯⋯⋯⋯⋯⋯⋯⋯⋯⋯⋯441
7.1.2常见错误与警告⋯⋯⋯⋯⋯⋯⋯⋯⋯⋯⋯⋯⋯⋯⋯⋯⋯⋯⋯⋯⋯⋯⋯⋯⋯⋯⋯⋯⋯⋯⋯⋯⋯⋯⋯⋯⋯⋯⋯⋯⋯⋯⋯⋯⋯⋯⋯⋯⋯⋯⋯⋯⋯⋯444
TEX错误/444·ETpX错误/448·TEX警告/451·LTEX警告/452
7.2调试与分析⋯⋯⋯⋯⋯⋯⋯⋯⋯⋯⋯⋯⋯⋯⋯⋯⋯⋯⋯⋯⋯⋯⋯⋯⋯⋯⋯⋯⋯⋯⋯⋯⋯⋯⋯⋯⋯⋯⋯⋯⋯⋯⋯⋯⋯⋯⋯⋯⋯⋯⋯⋯⋯⋯454
7.2.1 调试命令⋯⋯⋯⋯⋯⋯⋯⋯⋯⋯⋯⋯⋯⋯⋯⋯⋯⋯⋯⋯⋯⋯⋯⋯⋯⋯⋯⋯⋯⋯⋯⋯⋯⋯⋯⋯⋯⋯⋯⋯⋯⋯⋯⋯⋯⋯⋯⋯⋯⋯⋯⋯⋯⋯⋯⋯⋯⋯⋯⋯454
7.2.2更多调试工具⋯⋯⋯⋯⋯⋯⋯⋯⋯⋯⋯⋯⋯⋯⋯⋯⋯⋯⋯⋯⋯⋯⋯⋯⋯⋯⋯⋯⋯⋯⋯⋯⋯⋯⋯⋯⋯⋯⋯⋯⋯⋯⋯⋯⋯⋯⋯⋯⋯⋯⋯⋯⋯⋯456
7.3提问的智慧⋯⋯⋯⋯⋯⋯⋯⋯⋯⋯⋯⋯⋯⋯⋯⋯⋯⋯⋯⋯⋯⋯⋯⋯⋯⋯⋯⋯⋯⋯⋯⋯⋯⋯⋯⋯⋯⋯⋯⋯⋯⋯⋯⋯⋯⋯⋯⋯⋯⋯⋯⋯⋯⋯461
7.3.1提问之前⋯⋯⋯⋯⋯⋯⋯⋯⋯⋯⋯⋯⋯⋯⋯⋯⋯⋯⋯⋯⋯⋯⋯⋯⋯⋯⋯⋯⋯⋯⋯⋯⋯⋯⋯⋯⋯⋯⋯⋯⋯⋯⋯⋯⋯⋯⋯⋯⋯⋯⋯⋯⋯⋯⋯⋯⋯461
7.3.2最小工作示例⋯⋯⋯⋯⋯⋯⋯⋯⋯⋯⋯⋯⋯⋯⋯⋯⋯⋯⋯⋯⋯⋯⋯⋯⋯⋯⋯⋯⋯⋯⋯⋯⋯⋯⋯⋯⋯⋯⋯⋯⋯⋯⋯⋯⋯⋯⋯⋯⋯⋯⋯⋯⋯⋯⋯⋯462
7.3.3坏问题·好问题⋯⋯⋯⋯⋯⋯⋯⋯⋯⋯⋯⋯⋯⋯⋯⋯⋯⋯⋯⋯⋯⋯⋯⋯⋯⋯⋯⋯⋯⋯⋯⋯⋯⋯⋯⋯⋯⋯⋯⋯⋯⋯⋯⋯⋯⋯⋯⋯⋯⋯⋯⋯⋯⋯⋯⋯⋯465
本章注记⋯⋯⋯⋯⋯⋯⋯⋯⋯⋯⋯⋯⋯⋯⋯⋯⋯⋯⋯⋯⋯⋯⋯⋯⋯⋯⋯⋯⋯⋯⋯⋯⋯⋯⋯⋯⋯⋯⋯⋯⋯⋯⋯⋯⋯⋯⋯⋯⋯⋯⋯⋯⋯⋯⋯⋯⋯⋯⋯468
第8章 IATEX无极限⋯⋯⋯⋯⋯⋯⋯⋯⋯⋯⋯⋯⋯⋯⋯⋯⋯⋯⋯⋯⋯⋯⋯⋯⋯⋯⋯⋯⋯⋯⋯⋯⋯⋯⋯⋯⋯⋯⋯470
8.1宏编辑浅说⋯⋯⋯⋯⋯⋯⋯⋯⋯⋯⋯⋯⋯⋯⋯⋯⋯⋯⋯⋯⋯⋯⋯⋯⋯⋯⋯⋯⋯⋯⋯⋯⋯⋯⋯⋯⋯⋯⋯⋯⋯⋯⋯⋯⋯⋯⋯⋯⋯⋯⋯⋯⋯⋯471
8.1.1从ETEX到TEX ⋯⋯⋯⋯⋯⋯⋯⋯⋯⋯⋯⋯⋯⋯⋯⋯⋯⋯⋯⋯⋯⋯⋯⋯⋯⋯⋯⋯⋯⋯⋯⋯⋯⋯⋯⋯⋯⋯⋯⋯⋯⋯⋯⋯⋯⋯⋯⋯⋯⋯⋯⋯⋯⋯ 471
8.1.2编写自己的宏包和文档类⋯⋯⋯⋯⋯⋯⋯⋯⋯⋯⋯⋯⋯⋯⋯⋯⋯⋯⋯⋯⋯⋯⋯⋯⋯⋯⋯⋯⋯⋯⋯⋯⋯⋯⋯⋯⋯⋯⋯⋯⋯⋯⋯⋯⋯⋯478
8.2外部工具举隅⋯⋯⋯⋯⋯⋯⋯⋯⋯⋯⋯⋯⋯⋯⋯⋯⋯⋯⋯⋯⋯⋯⋯⋯⋯⋯⋯⋯⋯⋯⋯⋯⋯⋯⋯⋯⋯⋯⋯⋯⋯⋯⋯⋯⋯⋯⋯⋯⋯⋯⋯⋯483
8.2.1 自动代码生成⋯⋯⋯⋯⋯⋯⋯⋯⋯⋯⋯⋯⋯⋯⋯⋯⋯⋯⋯⋯⋯⋯⋯⋯⋯⋯⋯⋯⋯⋯⋯⋯⋯⋯⋯⋯⋯⋯⋯⋯⋯⋯⋯⋯⋯⋯⋯⋯⋯⋯⋯⋯⋯⋯⋯⋯⋯483
生成公式代码/483·生成表格代码/484·  生成图形代码/487·  生成完整的TEX文档/489
8.2.2在其他地方使用ビTEX⋯⋯⋯⋯⋯⋯⋯⋯⋯⋯⋯⋯⋯⋯⋯⋯⋯⋯⋯⋯⋯⋯⋯⋯⋯⋯⋯⋯⋯⋯⋯⋯⋯⋯⋯⋯⋯⋯⋯⋯⋯⋯⋯⋯⋯⋯⋯⋯⋯⋯492
8.3 LTEX资源寻找⋯⋯⋯⋯⋯⋯⋯⋯⋯⋯⋯⋯⋯⋯⋯⋯⋯⋯⋯⋯⋯⋯⋯⋯⋯⋯⋯⋯⋯⋯⋯⋯⋯⋯⋯⋯⋯⋯⋯⋯⋯⋯⋯⋯⋯⋯⋯⋯⋯⋯⋯⋯493
8.3.1 再探TEX发行版⋯⋯⋯⋯⋯⋯⋯⋯⋯⋯⋯⋯⋯⋯⋯⋯⋯⋯⋯⋯⋯⋯⋯⋯⋯⋯⋯⋯⋯⋯⋯⋯⋯⋯⋯⋯⋯⋯⋯⋯⋯⋯⋯⋯⋯⋯⋯⋯⋯⋯⋯⋯⋯⋯493
8.3.2互联网上的ETEX⋯⋯⋯⋯⋯⋯⋯⋯⋯⋯⋯⋯⋯⋯⋯⋯⋯⋯⋯⋯⋯⋯⋯⋯⋯⋯⋯⋯⋯⋯⋯⋯⋯⋯⋯⋯⋯⋯⋯⋯⋯⋯⋯⋯⋯⋯⋯⋯⋯⋯⋯496
CTAN/496·TEX用户组织/497·在线社区与独立网站/498
本章注记⋯⋯⋯⋯⋯⋯⋯⋯⋯⋯⋯⋯⋯⋯⋯⋯⋯⋯⋯⋯⋯⋯⋯⋯⋯⋯⋯⋯⋯⋯⋯⋯⋯⋯⋯⋯⋯⋯⋯⋯⋯⋯⋯⋯⋯⋯⋯⋯⋯⋯⋯⋯⋯⋯⋯⋯⋯⋯⋯501
xiv
部分习题答案⋯⋯⋯⋯⋯⋯⋯⋯⋯⋯⋯⋯⋯⋯⋯⋯⋯⋯⋯⋯⋯⋯⋯⋯⋯⋯⋯⋯⋯⋯⋯⋯⋯⋯⋯⋯⋯⋯⋯⋯⋯⋯⋯⋯⋯⋯⋯⋯502
参考文献⋯⋯⋯⋯⋯⋯⋯⋯⋯⋯⋯⋯⋯⋯⋯⋯⋯⋯⋯⋯⋯⋯⋯⋯⋯⋯⋯⋯⋯⋯⋯⋯⋯⋯⋯⋯⋯⋯⋯⋯⋯⋯⋯⋯⋯⋯⋯⋯⋯⋯⋯523
索引⋯⋯⋯⋯⋯⋯⋯⋯⋯⋯⋯⋯⋯⋯⋯⋯⋯⋯⋯⋯⋯⋯⋯⋯⋯⋯⋯⋯⋯⋯⋯⋯⋯⋯⋯⋯⋯⋯⋯⋯⋯⋯⋯⋯⋯⋯⋯⋯⋯⋯542

\section{熟悉 LaTeX}
\label{sec:熟悉 LaTeX}

\LaTeX 是一种基于 \TeX 的文档排版系统。
TEX只这么交错起伏的几个字母,便道出了“排版”二字的几分意味:精确、复杂、注重细节和品位。
而 \LaTeX 则为了减轻这种写作、排版一肩挑的负担,把大片排版的格式细节隐藏在若干样式之后,以内容的逻辑结构统帅纷繁的格式,遂成为现在最流行的科技写作———尤其是数学写作的工具之一。
无论你是因为心慕IATEX漂亮的输出结果,还是因为要写论文投稿被逼上梁山,都不得不面对一个事实:
IETEX是一种并不简单的计算机语言,不能只点点鼠标就弄好一篇漂亮的文章,也不是一两个小时的泛泛了解就尽能对付得过去的①。
还得拿出点上学搞研究时的那股钻研劲儿,才能通过手指下的键盘,编排出整齐漂亮的文章来。

①是的,有一个著名的入门教程就叫《112分钟学会KHEX》[187]。不过这个分钟其实是以页码计算的,粗粗浏览一遍还远算不上学会。而且即使掌握了这个教程中的内容,仍然可能在实际写作中遇到许多难以解决的问题。本书同样不打算让你能迅速变成一个高手。

LATEX的读音和写法
TEX一名源自technology的希腊词根τεχ, TEX之父高德纳教授②近乎固执地要求[126]它的发音必须是(按国际音标)[tεx],尽管英语中它常被读做[tεk]。(同样,高德纳教授也近乎固执地要求别人说他的姓Knuth时不要丢掉"K",叫他Ka-NOOTH,尽管在英语环境他时常会变成Nooth教授。)对比汉语,TEX的发音近似于“泰赫”,而且可以用汉语拼音准确地拼出来:têh(或许老一辈的人习惯用注音:去せ厂)。

②Donald Ervin Knuth, Stanford大学计算机程序设计艺术荣誉教授, Turing奖和von Newmann奖得主。高德纳是他的中文名字。TpX系统就是高德纳为了排版他的七卷本著作《计算机程序设计艺术》而编制的。

LETEX这个名字则是把ETEX之父Lamport博士①的姓和TpX混合得到的。所以IATEX大约应该读成“拉泰赫”。
不过人们仍然按着自己的理解和拼写发音习惯去读它:['lɑːtεk]、['leɪtεk]或是[lɑː'tεk], 甚至不怎么合理的 ['leɪtεks]。
好在Lamport并不介意ETEX到底被读做什么。
“读音最好由习惯决定,而不是法令。”————Lamport如是说[136,§1.3].

两个创始人对于名称和读音的不同态度或许多少说明了这样一个事实:
ETEX相对原始的TEX更少关注排版的细节,因此ETEX在很多时候并不充当专业排版软件的角色,而只是一个文档编写工具。而当人们在IATEX中也抱以追求完美的态度并用到一些平时不大使用的命令时,通常总说这是在TEX层面排版————尽管IATEX本身正是运行于TEX之上的。
类似地,TEX和ETEX字母错位的排印也体现出一种面向排版的专业态度,即使在字符难以错位的场合,也应该按大小写交错写成TeX和LaTeX.

现在我们使用的ETEX格式版本为28,意思是超出了第2版,接近却没有达到第3版,因此写成IATEX2E。
在只能使用普通字符的场合,一般写成LaTeX2e.

\subsection{让 \LaTeX 跑起来}%
\label{sub:让LaTeX跑起来}

学习ETEX的第一步就是上手试一试,让ETEX跑起来。
首先安装TEX系统及其他一些必要的软件,然后跑一个测试的例子。
下面的几节包含了一大堆具体软件安装和使用的内容,虽然有些烦琐,但这是使用BTEX进行写作的必要前提。
如果你早已做好这些准备,或者在读本书以前就已经迫不及待地做了不少尝试的话,可以直接跳到第32页1.2节开始第一个实际规模的例子。

\subsubsection{让 \LaTeX 跑起来}%
\label{sub:让LaTeX跑起来}

1.1.1 IATEX的发行版及其安装
TEX/ETEX并不是单独的程序,现在的TEX系统都是复杂的软件包,里面包含各种排版的引擎、编译脚本、格式转换工具、管理界面、配置文件、支持工具、字体及数以千计的宏包和文档。一个TEX发行版(Distribution)就是把所有这样的部件都集合起来,打包发布的软件。
尽管内容庞杂,但现在的TEX发行版的安装还是非常方便的。下面将介绍两个最为流行的发行版,一是1.1.1.1节的CTEX套装,二是1.1.1.2节的TEX Live.前者是
①Leslie Lamport博士,微软研究院资深研究员, Dijkstra奖得主。
1
第1章	熟悉ETEX	3
Windows系统下的软件,后者则可以用在各种常见的桌面操作系统上。对Windows用户来说,两个发行版并没有显著的优劣之分,你可以任选一个安装使用。
请注意:下面介绍的发行版都是在写作本书时最新的版本。然而当你读到这一段时,软件可能已经更新,界面也可能会有些不同。不过不用担心,安装的过程和使用方法大体上都是一样的。
1.1.1.1 CTEX套装
CTEX套装是由中国科学院的吴凌云制作并维护的一个面向中文用户的Windows系统下的发行版。这个发行版事实上是对另一个发行版MiKTEX的再包装,除了MiKTEX主体以外, CTEX套装增加了WinEdt作为主要编辑器,以及PDF预览器SumatraPDF,PostScript文件预览器GSview, PostScript解释器GhostScript, 一些旧的中文支持包和工具(如CCT系统)和其他一些有关中文的额外配置(如额外中文字体配置)。
CTEX套装或许是中文ETEX用户最常用的发行版了。它一直以安装简单、容易上手著称。CTEX套装有基本版和完全版之分,基本版只包含一些基本安装的MiKTEX系统,实际使用中缺少的宏包会在编译时自动下载安装,或由用户自己选择手工安装;而完全版则包含了完整的MiKTEX所有组件。对于一般用户,建议使用完全版的CIEX套装,这不仅避免了编译时因缺少宏包还要临时下载的问题,而且完全版中包含的诸多文档资料对于用户也很有用。只要从http://www.ctex.org/CTeXDownload下载对应版本的安装文件,就可以直接进行安装,见图1.1.
CTEX套装安装好后,会在“开始”菜单增加一个项目,里面有多个子项目。其中WinEdt①和MiKTeX目录下的TeXworks是最主要的ETEX编辑器,多数时间我们都将在这两个编辑器之中工作。如果你已经完成安装,现在就可以跳到第13页1.1.2节开始熟悉使用编辑器了。
“开始”菜单中的其他项目也值得注意。
FontSetup
为CTEX套装重新安装CJK宏包使用的中文字体。CTEX套装使用Windows操作系统所安装的中文字体进行配置,默认支持宋体、黑体、仿宋、楷体、隶书、幼圆6种,其中前4种是中文版Windows预装的字体,后两种是中文版Office系统预装的字体。如果系统没有安装对应的字体,则不能进行配置安装。
08Uninstall CTeX
卸载CTEX套装。
①WinEdt是商业共享软件,用户可以免费试用一个月。
1
4	1.1	让ETEX跑起来
CTeX 2.9.2安装
CTeX 2.9.2安装
选择组件
欢迎使用“CTeX 2.9.2”安装向导
选择你想要安装“CTeX 2.9.2”的那些功能。
这个向导将指引你完成“CTeX 2.9.2”的安装进程。
勾选你想要安装的组件,并解除勾选你不希望安装的组件。单击[下一步00]继
省防伪高器群策”更新指定的系统文件,而不需要重新
选定安装的组件:
描述
1170
单击 [下一步00]继续。
CTeX Addons
Ghostscript
GSview
WinEdt
所需空间:577.2MB
下一步(N)
取消(C)
<上一步②
下一步(N)
取消(C)
CTeX 2.9.2安装
CTeX 2.9.2 安装
选择安装位置
正在安装
选择“CTaX 2.9.2”的安装文件夹。
"CTeX 2.92”正在安装,请等候。
抽取:hycolor. pdf
抽取:example-syncolorsetup. sty
抽取:flags. pdf
抽取:gettitlestring pdf
抽取:grfext. pdf
目标文件夹
抽取:profile. pdf
抽取:hobsub. pdf
浏览①…
抽取:hologo. pdf
抽取:holtxdoc. pdf
所需空间:577.2MB
抽取:hopatch. pdf
可用空间:4.9GB
抽取:hycolor. pdf
CTaX 2.9.2.163 C) CTEX.08G
上一步(P)
安装(I)
取消(C)
<上一步①下一步④>
取消①
图1.1 在Windows7中安装CTEX 套装2.9
GhostScript
/
GhostScript程序是PostScript的解释器,许多TEX程序都依赖它工作。在命令行下经常还可以使用它转换一些图像格式。
Ghostgum
这个目录里面是PostScript文件. ps和. eps的查看工具GSview①, 类似于TEX Live中的PS View.也可以用它来查看PDF文件,不过效果没有Adobe Reader好.安装后. ps和. eps文件会与这个程序关联。
Help
里面是一些由CTEX套装所附带的额外的帮助文档。包括一个常见问题集[[308](CTeX FAQ)、《IETEX28插图指南》[204](Graphics)、一个入门文档lshort[187]
__
①GSview是一个发布于AFPL协议下的开源免费软件,运行时可能会有注册的弹窗,但软件本身是无须注册的,不影响使用。
1
第1章	熟悉ETEX	5
(LaTeX Short)、一个ETEX参考手册[23](LaTeX2e Reference Manual)、《ETEX Companion》第八章数学公式部分[166] (Mathematics)、一份符号大全[192] (Symbols)和英文的常见文题集[270] (UK TeX FAQ).
不过遗憾的是,这里提供的部分资料有些陈旧。CTEX的常见问题集已经几年没有更新,关于中文处理的内容大大落后于现在的实际情况;《ETEX28插图指南》也是翻译自几年前的文档,个别内容已经有所变化。本书涵盖了上面内容文档中除符号表外的大多数内容。但无论如何,这里选取的几个文档可以说是日常使用中最实用的一些,还是值得一看的。
MiKTeX
MiKTEX目录下有好几个项目。Previewer是MiKTEX的DVI文件预览器,叫做Yap, 类似于TEX Live中的DVIOUT, 不过我们很少会用它; TeXworks是一个小巧好用的编辑器; Help目录下是MiKTEX这个发行版本身的文档; Maintenance和Maintenance(Admin)目录中是MiKTEX的对Windows当前用户和所有用户的配置工具;而MiKTeX on the Web目录中则是MiKTEX网站的快捷方式。
这里需要详细说明的是MiKTEX的配置工具(Maintenance, 如图1.2所示)。其中有三项:Package Manager是MiKTEX的包管理工具; Settings将打开MiKTEX的配置选项MiKTeX Options; 而Update则是MiKTEX的在线升级程序。
Package Manager
利用包管理器(Package Manager)可以查看和检索MiKTEX共有哪些宏包,已经安装了哪些宏包,也可以在线安装和删除各种宏包。所有宏包都有一个简单的介绍和分类,对于喜欢刨根问底,打算了解自己计算机上到底安装了什么东西的人来说,包管理器是一个很好的切入口。如果要安装新宏包,请注意首先选好MiKTEX的软件仓库(Repository)并进行同步(Synchronize)。软件仓库通常选取一个CTAN网站镜像的MiKTEX目录,如CTEX网站的镜像。
18Settings
MiKTEX选项(MiKTeX Options)里面是一些关于MiKTEX发行版的整体配置。
在General选项卡中,可以刷新文件名数据库(Refresh FNDB)或更新格式(Update Formats),这通常用在手工安装或更新了宏包和工具的时候;可以设置默认的纸张大小;也可以设置在编译时缺少宏包时是不是自动在线安装(这是MiKTEX系统的特色功能)。
在Roots选项卡中,可以查看、改变或增加TEX的根目录。每个TEX根
1
6
1.1 让EIEX跑起来
MiKTeX Package Manager
Edit	View	IaskRepository Help
Name:	Keywords:	Fle name:	Filter	Reset
Name	Category	Size	Packaged onInstalled on	Title
metago	\Applications\Graphics	90224	2008-08-28	MetaPost outpu
metalogo	\Formats\LaTeX\LaTeX contrib	110210	2009-09-11	Extended TeX k
metaobj	\Applications\Graphics	1322759	2007-08-27	MetaPost pack:
metaplot	\Uncategorized	334773	2005-06-25
metapost-exam. .. \Documentation	128727	2001-10-03	Example drawir
metatex	\Formats\LaTeX\LaTeX contrib	218867	2004-08-15	METATeX comr
metauml	\Uncategorized	701259	2006-03-25	MetaPost librar
method	\Formats\LaTeX\LaTeX contrib	46026	2001-05-14	Typeset methoc
mex	\Language Support\Polish	194278	2006-05-19	A Polish format
mf-ps	\Applications\Graphics	155559	2001-05-14	A MetaFont-Por
mff	\Formats\LaTeX\LaTeX contrib	300098	2001-05-14
mflogo	\Fonts\METAFONT Fonts	53183	2001-05-14	2009-09-12	LaTeX support
mfnfss	\Formats\LaTeX\LaTeX contrib	70253	2001-05-14
mfpic	\Applications\Graphics	2211195	2009-11-27	Draw MetaFont,
mfpic4ode	\Formats\LaTeX\LaTeX contrib	579177	2009-04-21	Macros to drav
Total:1798
(a)包管理器(Package Manager)
MiKTeX Options
✗
General
Roots
Formats
Languages
Packages
基 aintensance
Update MiKTeX
Refresh the file name databasewhenever you install or renove
Eefresh FNDB
Update Source
MiK
Update all forast files when youhave installed now packages.
Update Fornats
TEX
Paper
I want to get updated packages from a remote package responsivery
Select your default paper
A4 (A4size)
Last used remote package repository
Let me choose a remote package repository.
Package installation
Connection Settings..
installed on-the-fly.
I want to get updated packages from a local package repository:
Install missing packages
Ask me firs▼
Last used directory location.
Let" : specify a directory location.
I want to get updated packages from a MiKTeX CD/DVD.
确定
取消
应用(A)
上一步⑧
下一步(N)
取消
(b)选项设置(Options)
(c)更新
图1.2 MiKTEX配置工具
1
第1章	熟悉ETEX	7
目录下的目录树结构都是基本相同的,只有按照这种结构放置的文件才能被正确找到并使用。这种树结构一般称为TDS结构(TEX Directory Structure,参见[269])。一般用户自己编写的文件和一些从第三方得到的宏包、字体、文档等,都放在单独的TDS根目录中,在CTEX套装中安装目录下的CTEX目录就是这样一个TDS根目录。
2②
Formats选项卡用来管理TEX系统的编译格式。TEX和相关的宏语言可能有多种格式(format),INITEX等程序为每个格式以预编译的方式生成一些二进制格式的信息,并与对应的编译命令(如pdflatex、mpost等)结合起来。一般没有必要修改这里的内容。
Language选项卡可以管理一些语言(不包括中文,主要是西方语言)的支持文件。Packages选项卡与包管理器的功能类似。可以查看和修改已安装的MiKTEX包。
88Update MiKTeX
这是MiKTEX的升级程序,可以用于更新宏包或升级整个MiKTEX系统。CTEX套装的主体就是MiKTEX,因此可以不重装CTEX套装,直接使用MiKTEX的升级程序完成除旧式中文支持和编辑器配置外的大部分升级工作。
1.1.1.2 TEX Live
TEX Live是由TUG(TEX User Group, TEX用户组)发布的一个发行版。TEX Live可以在类UNIX/ Linux、Mac OS X和Windows等不同的操作系统平台下安装使用,并且提供相当可靠的工作环境①。TEX Live可以安装到硬盘上运行,也可以经过便携(portable)方式安装刻录在光盘上直接运行(故有“Live”之称)。
有两种安装TEX Live的方式:一是从TEX Live光盘进行安装,二是从网络在线安装。不同操作系统下安装设置TEX Live的方式基本一样,这里仍以Windows操作系统为例进行演示。
一、从光盘安装
TEX Live一般以安装光盘镜像的方式在互联网上发布。光盘镜像文件可以从TUG②或CTAN③网站上下载。可以把镜像文件刻录到DVD光盘上使用,也可以直接加载到
①例如在中文支持方面,旧版本MiKTpX的中文字体配置一直有一些错误,所以CTEX套装做了进一步配置才正确支持中文;而TEX Live就没有这种问题。
②http://www.tug.org/texlive/
③CTAN有很多镜像网站,参见8.3.2节,国内常用的镜像是CTEX网站的FTP镜像:ftp://ftp.ctex.org/mirrors/CTAN/ systems/ textive/ Images/.
1
8	1.1 让ETEX跑起来
虚拟光驱上进行安装。
装入光盘后,安装程序会自动运行(见图1.3)。如果系统禁用了自动运行,可以手动执行光盘根目录的install-tl. bat安装。只要选择好安装的位置,不断单击“下一步”按钮就可以安装TEX Live了。
74 Install-tl
TeX Live 2012安装
2/5
符包含整个安装的目的文件夹。
强烈建议以年份作为最后的目录项。
目的文件夹:
C:\textive\2012
修改
需要的磁盘空间3246MB
退出
(上一步
下一步>
图 1.3 在Windows 7下安装TEX Live 2012
如果对TEX系统已经比较熟悉,还可以运行光盘根目录的install-tl-advanced. bat进行可定制安装(见图1.4)。此时,除了安装的位置以外,还可以从预置的几种安装方案中选择某种进行安装,可以选择安装的语言、宏包、工具、文档集合,或进行进一步的安装配置。例如,如果要在服务器上安装后台服务,不想让TEX系统占用太大的空间,可以去掉所有的文档和源代码,只选择安装少量必需的宏包和工具,只用原来几分之一的硬盘空间安装一份基本可用的系统。
对于Linux系统的用户,还需要设置环境变量并为XgTEX配置字体。设置Linux环境变量的方式参见[25,§3.4],我建议偷懒的用户在安装时选择在标准路径下创建符号链接的选项,这样就不必设置环境变量了。下面则需要为XgTEX配置字体,让操作系统的fontconfig库能找到TEX Live附带的字体,按下面步骤操作:
1. 进入TEX Live的TEXMFSYSVAR/ fonts/ conf/目录(其中TEXMFSYSVAR是一个变量,在定制安装时选定。其默认值为/ usr/ local/ texlive/2012/ texmf-var/),将里面的texlive-fontconfig. conf文件改名为09-texlive. conf, 复制到/ etc/ fonts/ conf. d/目录。可以在命令行下(参见1.1.2.3节)执行命令:
1
第1章	熟悉ETEX	9
✗
74 Install-tl
TeX Live 2012安装
——基本信息——
选择安装方案
scheme-full
修改
……进一步定制……
标准安装
修改
语言集合
修改
85集合来自 85(需要的磁盘空间3246 MB)
——目录设置——
Portable setup
否
切换
TEXDIR(主 TeX目录)
C:\textive\2012
修改
TEXMFLOCAL(存放本地格式文件等)
C:\textive\texmi-local
修改
TEXMFSYSVAR(存放自动生成数据的目录)
C:\textive\2012\texmf-var
修改
TEXMFSYSCONFIG(存放本地配置)
C:\texdive\2012\texmi-config
修改
TEXMFHOME(用户专有文件的目录)
~\text/
修改
----选项-----
缺省的纸张给
A4
切换
允许用\write18执行一部分在限制列表内的程序
是
切换
创建所有格式文件
是
切换
安装字体/宏包文档日录树
是
切换
安装字体/宏包源码目录树
是
切换
修改注册表中的 PATH设置
是
切换
Add menu shortcuts
是
切换
修改文件关联
只有新的
修改
安装 TeXwoks前端
是
切换
After installation, get package updates from CTAN
否
切换
v27178/26745
安装 TeXLive
退出
图1.4 定制安装 TEX Live 2012
sudo cp / usr/ local/ texlive/2012/ texmf-var/ etc/ fonts/ texlive-fontconfig. conf \/ etc/ fonts/ conf. d/09-texlive. conf
2. 刷新fontconfig的字体缓存,执行命令:
sudo fc-cache -fsv
如果一切正常,你会看到屏幕上提示缓存了TEX Live一些目录中的字体。
这一配置过程也将使你可以在其他程序中使用TEX Live所安装的几百种字体。在类UNIX系统下安装TEX Live的过程比在Windows下略显复杂,希望这个情况在以后能有所改观。
此外,如果希望pdfTEX、dvipdfmx等程序能正确找到操作系统中安装的字体,或让XqTEX能按字体文件名找到系统字体,还需要设置正确的OSFONTDIR变量。TEX Live会对Windows系统自动设置这一变量,对Linux等系统也需要手工修改。新建或修改在TEX Live安装目录(如/ usr/ local/ texlive/2012/)下的配置文件texmf. cnf,在里面修改OSFONTDIR变量的值,典型的值如:
OSFONTDIR = / usr/ share/ fonts//;/ usr/ local/ share/ fonts//;~/. fonts//
1
10	1.1 让ETEX跑起来
程序安装好后,会在桌面上增加TEX编辑器TeXworks和PostScript文件查看工具PS View的图标①,现在就可以进行工作了。
TEX Live的开始菜单相对简单。它包括以下项目:
TeXworks editor
这是TEX Live预装的一个的TEX文件编辑器,简单方便。大部分工作都可以在这个编辑器中完成。
DVIOUT DVI viewer
这是一个DVI文件预览器,类似于MiKTEX中的Yap.不过我们很少会用到它。
PS View
这是PostScript文件查看工具,和桌面上的图标一样。也可以用它来查看PDF文件,不过效果没有Adobe Reader好。安装后 . ps和. eps文件会与它关联。
TeX Live command line
它打开Windows的命令提示符(参见1.1.2.3节),并设置好必要的环境变量,可以在其中使用命令行编译处理TEX文档。
TeX Live documentation
这是一个HTML页面的链接,里面是TEX Live系统中所有PDF或HTML格式的文档列表。在首页你可以找到几种语言(包括简体中文)的TEX Live发行版文档,以及到近2000份各种文档的列表的链接———这份有一公里长的列表多少说明了TEX Live是一个多么复杂的系统,以及它在完全安装时为什么占用了这么大的空间。当然,你不需要读完里面的所有文档才能学会使用IATEX,不过你会发现在工作中总需要时不时地查看里面的东西(参见8.3.1节)。
TeXdoc GUI
这是一个常用文档的列表,不过以图形界面的方式把文档分成若干类别,还可以搜索(见图1.5)。这里面直接列出的宏包数量较少,用来简单浏览可以,但如果要查看更多的内容,最好使用其文件搜索功能或利用命令行texdoc工具(参见8.3.1节)。
153	TeX Live Manager
这是TEX Live管理工具的图形界面(见图1.6),简称tlmgr.管理工具也可以在命令行下用tlmgr命令运行,用tlmgr gui可以在命令行下打开图形界面。
①TeXworks和PS View是在Windows下安装的附加软件。在其他操作系统如Linux中,通常都已经安装或容易从其他途径安装类似的软件,如Kile和Evince.
1
第1章	熟悉ETEX	11
Tk TeX Documentation Browser
QuitDatabase searchFile searchgeometry	SettingsHelp/ About
Guides and tutorials	Diagrams	Auxiliary tools
Fundamentals	Slides	Education
Macro programming	Tables, arrays and lists	Tex on the Web
Accessory programs	ToC, index and glossary	Extended Systems
Fonts/ Metafont	Bibliography	The TeX Live Guide
Languages/ national specials	Mathematics	Music
General layout	Special text elements	Compuscripts
Floats	Typesetting labels	Games
Graphics	Verbatim and code printing	Miscellaneous
图1.5 TeXdoc GUI
Tk TeX Live Manager 2012
tlmgr
选项
操作
帮助
已载入的软件包仓库
http://ftp.ctex.org/mirrors/CTAN/systems/textive/tlnet/
显示配置
状态
分类
匹配
选择
全部的
软件包
全部的
全部选择
已安装
集合
简短描述
选定的
全部不选
未安装
套装
taxonomies
未选的
更新
filenames
重置过滤器
软件包名称
本地版本
远程版本
简短描述
scheme-basic
25923
25923
basic scher
scheme-context
26699
26699
ConTeXt sc
scheme-full
21417
21417
full scheme
更新全部已安装的
更新
安装
删除
备份
重装先前删除的包
…… done loading.
tlmgr: package repository
http://ftp.ctex.org/mirrors/CTAN/systems/texlive/tl
net
/
图1.6 TEX Live Manager (TEX Live管理工具)
1
12	1.1 让ETEX跑起来
可以用tlmgr从网络上或光盘中安装、删除或更新宏包及组件,在开始安装或更新组件前,注意选择正确的软件包仓库(光盘目录或CTAN上的目录)并载入。
也可以在菜单中进行一些其他的配置。在“操作”菜单中,“更新文件名数据库”就是运行texhash程序,如果手工安装宏包(未使用tlmgr),就需要执行这个操作;“重新创建所有格式文件”就是运行fmtutil程序,如果手工更新了一些程序,需要执行这个操作;“更新字体映射数据库”则对应于updmap程序,如果手工安装了PostScript字体(如一些商用字体),则需要执行这个操作。
TEX Live较新版本的tlmgr程序的图形界面可能与上面描述的有所不同,但配置的内容和操作方法基本是一致的。如果还有疑问,可参阅TEX Live的手册[25].
对Linux用户来说,Linux发行版的软件源也可能会将TEX Live另行打包,以方便通过Linux的软件源安装,例如Ubuntu Linux的软件源里面就有若干以texlive开头的apt包.操作系统自带的TEX Live往往比较陈旧或被分割简化,特别是难以利用CTAN源更新,不过好处是安装起来更容易些。我建议最好还是自己安装[25]。许多Linux软件依赖TEX系统(如TEX编辑器Kile),在安装时要求先安装操作系统的texlive包,与自己安装的TEX Live发行冲突。解决这类包依赖问题可以使用虚拟包(dummy package),或在手动下载相关包手在命令行下强制安装,或直接从源代码安装依赖TEX Live的软件,不过这方面的内容已经超出了本书的范围,你可以在你的Linux发行版的社区请教相关的专家。
二、从网络安装
也可以从网络上在线安装TEX Live系统。这样可以保证安装的组件都是最新版本,而且如果进行定制安装,就只需要下载需要的部分,节省下载时间。
网络安装需要先从CTAN镜像的systems/ texlive/ tlnet/目录下载安装工具。如CTEX网站的CTAN镜像(参见8.3.2节):
http://ftp.ctex.org/mirrors/CTAN/systems/texlive/tlnet/
下载对应操作系统的install-tl安装脚本:Windows用户下载install-tl. zip, Linux和其他类UNIX用户下载install-tl. tar. gz.
从下载的压缩包解压得到安装工具后,安装过程与在光盘上安装完全一样。Windows用户只要双击执行解压出的install-tl. bat或 install-tl-advanced. bat就会出现图1.3或图1.4的安装界面了,按提示进行安装,程序会自动从网络上下载所需的文件进行安装。如果网络比较的话,用这种方式安装不比用光盘安装慢多少。
1
第1章	熟悉ETEX	13
默认情况下,安装程序会自动选择较近的CTAN镜像服务器,不过教育网用户可能不方便访问国外的网站,需要在命令行手工指定国内的CTAN镜像服务器地址。例如运行如下命令从CTEX网站安装TEX Live:
install-tl -repositoryhttp://ftp.ctex.org/mirrors/CTAN/systems/texlive/tlnet/
1.1.2  编辑器与周边工具
1.1.2.1 编辑器举例———TeXworks
像其他计算机语言一样,ATEX使用纯文本描述,因而任何能编辑纯文本的编辑器都能编辑ETEX文档,如Windows系统的记事本、写字板, Linux下的VI、GEdit.不过,使用专门为IATEX设计或配置的编辑器,进行语法高亮、命令补全、信息提示、文档排版等工作,会使工作方便许多。
IATEX代码编辑器有很多,大致可以分为两类:一是主要为TEX/IATEX代码编辑而专门设计的编辑器,二是可以为TEX/ETEX代码编辑配置或安装插件的通用代码编辑器。前者如WinEdt、TeXworks、TeXMaker、Kile, 后者如Emacs、VIM、Eclipse、SciTE等.通常前一种编辑器配置和使用更简单些,下面主要以TeXworks为例说明编辑器的一些简单配置。其他大部分编辑器在基本功能和设置上都大同小异,不难举一反三。
TeXworks是MiKTEX和Windows系统下TEX Live预装的编辑器,也是国际TEX用户组(TUG)发布并推荐的入门级编辑器。Linux系统下TEX Live没有自动安装TeXworks编辑器,你可以到TeXworks的网站①自己下载安装。
TeXworks的界面非常简洁(见图1.7):它分为两部分,左侧是TEX源文件的编辑器窗口,右侧是生成的PDF文件的预览窗口。左边的编辑器窗口最上面是标题栏和标准菜单项,接着是工具栏,中间最大的编辑区,最下面则是显示行列号的状态栏。右边的预览窗口把编辑区换成了PDF预览区。
除了文本编辑区,编辑器窗口中最常用的是工具栏。工具栏的最左边的按钮是整个编辑器最为重要的“排版”按钮,它调用具体的命令把输入的TEX源文件编译为对应的PDF结果,刷新右边PDF文件的显示。紧靠排版按钮右边的下拉菜单用来选择排版时所使用的命令,通常对应一条单一的命令(如TEX Live中的版本或自己单独下载安装的版本),但也可以配置为好几条复合命令(如CTEX套装或纯MiKTEX中的版本)。通常我们使用最多的排版命令是“XeLaTeX”或“PDFLaTeX”,视具体情况而定。使用排版按钮时,未保存的文档会自动保存。工具栏剩下的按钮则是一系列常见的标准按钮:新建、打开、保存;撤销、重做;剪切、复制、粘贴;查找和替换,不必多说。
①http://code.google.com/p/texworks/

%\bibliography{}
\end{document}
