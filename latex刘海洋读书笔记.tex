%-*- coding: UTF-8 -*-
% 文件名:LaTeX使用手则.tex
% 描述:
% 指导我怎么规范使用 LaTeX
\documentclass[UTF8]{ctexart} % 使用中文文档类

%%%%%%%%%%%%%%%%%%%%%%%%%%%%%%%%%%%%%%%%%%%%%%%%%%%%%%%%%%%%%%%%%%%%%%%
%                               package                               %
%%%%%%%%%%%%%%%%%%%%%%%%%%%%%%%%%%%%%%%%%%%%%%%%%%%%%%%%%%%%%%%%%%%%%%%

%%%%%%%%%%
%  作者  %
%%%%%%%%%%
\usepackage{authblk} % 多作者
\renewcommand\Authfont{\normalsize}      % 作者名字保持正常字号
\renewcommand\Affilfont{\scriptsize}     % 机构名称使用较小字号

%%%%%%%%%%%%
%  超链接  %
%%%%%%%%%%%%
\usepackage{hyperref} % 网页超链接

%%%%%%%%%%%%%%%%%%
%  实现对勾和叉  %
%%%%%%%%%%%%%%%%%%
\usepackage{pifont} % 对勾和叉
\newcommand{\cmark}{\ding{51}} % 对勾
\newcommand{\xmark}{\ding{55}} % 错误标记



\usepackage{graphicx} % 插入图片
\usepackage{xcolor} % 启用颜色支持
\usepackage{float} % 使得表格不浮动

%%%%%%%%%%%%%%
%  verbatim  %
%%%%%%%%%%%%%%

\usepackage{fancyvrb} % 2.2.1节为了将代码放进盒子里,功能强大
\usepackage{cprotect} % 2.2.1节为了将代码放进盒子里
\usepackage{shortvrb} % 2.2.1节为了简写verb命令
%\MakeShortVerb|
%\DeleteShortVerb|
\usepackage{listings} % 代码高亮

%%%%%%%%%%
%  数学  %
%%%%%%%%%%
\newtheorem{thm}{定理}


%%%%%%%%%%%%%%
%  参考文献  %
%%%%%%%%%%%%%%
\bibliographystyle{plain}

%%%%%%%%%
%  mkt  %
%%%%%%%%%
\title{\LaTeX 编辑手册}
\author[1,2]{Lan \footnote{Email: lanalters@mail.ustc.edu.cn, Student ID: SA24168214}}
\affil[1]{中国科学技术大学, 合肥 230026}
\affil[2]{中国科学院合肥物质科学研究院\,等离子体物理研究所, 合肥 230031}
\date{\today}





%%%%%%%%%%%%%%%%%%%%%%%%%%%%%%%%%%%%%%%%%%%%%%%%%%%%%%%%%%%%%%%%%%%%%%%
%                               content                               %
%%%%%%%%%%%%%%%%%%%%%%%%%%%%%%%%%%%%%%%%%%%%%%%%%%%%%%%%%%%%%%%%%%%%%%%
\begin{document}

\maketitle
\tableofcontents


首要问题是我的 \LaTeX 文档里需要包含什么?
采用什么样的格式?
什么样的规范?
\LaTeX 的原则是内容和格式分离,因此它是极好进行预处理的格式。

\section{熟悉 LaTeX}
\label{sec:熟悉LaTeX}

\paragraph{\LaTeX 相对于 \TeX 的先进之处}

\LaTeX 则为了减轻写作、排版一肩挑的负担,把大片排版的格式细节隐藏在若干样式之后,以内容的逻辑结构统帅纷繁的格式,遂成为现在最流行的科技写作———尤其是数学写作的工具之一。
\footnote{因此要怎么排版,取决于内容的逻辑}

\subsection{让 \LaTeX 跑起来}
\label{sub:让LaTeX跑起来}

学习 \LaTeX 的第一步就是上手试一试,让 \LaTeX 跑起来。

\subsubsection{\LaTeX 发行版及其安装}
\label{subsub:LaTeX发行版及其安装}

程序安装好后,会在桌面上增加 \TeX 编辑器 \verb|TeXworks| 和 \verb|PostScript| 文件查看工具 \verb|PS_View| 的图标,现在就可以进行工作了。
但是似乎我安装的 \TeX Live 并没有安装 \verb|PS_View|。

\paragraph{TeX Live Manager}
\label{par:TeX Live Manager}

这是 \TeX Live 管理工具的图形界面(见图1.6),简称 \verb|tlmgr|。
管理工具也可以在命令行下用 \verb|tlmgr| 命令运行,用 \verb|tlmgr gui| 可以在命令行下打开图形界面。

\subsection{从一个例子说起}
\label{sub:从一个例子说起}

\subsubsection{确定目标}
\label{subsub:确定目标}

现在来把话题限定在初等平面几何,假定我们要写一篇关于勾股定理的短文,短文是一般的科技论文的模式,结构上包括标题、摘要、目录、几节的正文和最后的参考文献;内容包括文字、公式、图形、表格等。
分别来关注,这些部分如何实现。
此外,我还关心怎么把短文的页面设置得很小,四页拼成一页。

首先它有一些好习惯,比如前三行注释,第 1 行表明了这个文件的编码是 \verb|UTF-8|,这对中文文档往往非常有用;
第 2 行是源文件的文件名 \verb|gougu.tex|;
第 3 行则说明了源文件的内容:
\begin{verbatim}
%-* - coding: UTF-8- *-
% gougu. tex
% 勾股定理
\end{verbatim}
第 4 行是文档类,因为是中文的短文,所以使用 \verb|ctexart|,并用 \verb|[UTF8]| 选项说明编码。

\subsubsection{填写正文}
\label{subsub:填写正文}

\begin{itemize}
\item
  使用空行分段。
\item
  单个换行并不会使文字另起一段,而只是起到使源代码更易读的作用(上面的代码每行 35 个汉字)。
\item
  段前不用打空格,\LaTeX 会自动完成文字的缩进。
  即使手工在前面打了空格,\LaTeX 也会将其忽略,事实上它会忽略每行开始的所有空格。
  也不要使用全角的汉字空格,这通常会使排版的效果变得糟糕。
\item
  通常汉字后面的空格会被忽略,其他符号后面的空格则保留。
\item
  使用 \verb|xelatex| 编译文档时,\verb|ctexart| 文档类会调用 \verb|xeCJK| 宏包,自动处理汉字与其他符号之间的距离,无论你有没有在它们之间加上正确的空格,这是十分方便的。
  不过,在源代码中仍然可以给汉字与其他符号之间加上一个空格,这会令代码更加清晰。
\end{itemize}

\subsubsection{命令与环境}
\label{subsub:命令与环境}

继续排版短文的第 1 节,我们来处理脚注和引用内容。
脚注用脚注命令 \verb|\footnote| 得到的。
使用 \verb|\emph| 命令改变字体形状,表示强调(emphasis)的内容。
引用的内容则是在正文中使用 \verb|quote| 环境得到的。
不过,如果只使用 \verb|quote| 环境,并不能达到预想的效果:\verb|quote| 环境并不改变引用内容的字体。
因此还需要再使用改变字体的命令,即:
\begin{verbatim}
\begin{quote}
\zihao{-5}\kaishu 引用的内容。
\end{quote}
\end{verbatim}
文章的摘要也是在 \verb|\maketitle| 之后用 \verb|abstract| 环境生成的:
\begin{verbatim}
\begin{abstract}
  摘要
\end{abstract}
\end{verbatim}

使用定理,要先在导言区定义:
\begin{verbatim}
\newtheorem{thm}{定理}
\end{verbatim}
然后这样使用:
\begin{verbatim}
\begin{thm}[我的定理]
  定理内容
\end{thm}
\end{verbatim}

最后来注意一个小细节,前面在表示起迄年份时,用了两个减号 \verb|--|,这在 \LaTeX 中将输出一个“en dash”,即宽度与字母“n”相当的短线,通常用来表示数字的范围。

\subsubsection{遭遇数学公式}
\label{subsub:遭遇数学公式}

现在来看我们最关心的问题————输入数学公式,这大概是多数使用 \LaTeX 的人花费精力最多的地方了。
最简单的输入公式的办法是把公式用一对美元符号 \verb|$$| 括起来。
这种夹在行文中的公式称为“正文公式”(in-text formula)或“行内公式”(inline formula)。
对比较长或比较重要的公式,一般则单独居中写在一行;为了方便引用,经常还给公式编号。
这种公式被称作“显示公式”或“列表公式”(displayed formula),使用 \verb|equation| 环境就可以方便地输入这种公式:
\begin{verbatim}
\begin{equation}
  a(b+c) = ab + ac
\end{equation}
\end{verbatim}

\subsubsection{使用图表}
\label{subsub:使用图表}

首先来看插图。
在 \LaTeX 中使用插图有两种途径,一是插入事先准备好的图片,二是使用 \LaTeX 代码直接在文档中画图。
大部分情况下都是使用插入外部图片的方式,只在一些特别的情况大量用代码作图(如数学的交换图)。
插图功能不是由 \LaTeX 的内核直接提供,而是由 \verb|graphicx| 宏包提供的。
要使用 \verb|graphicx| 宏包的插图功能,需要在源文件的导言区使用 \verb|\usepackage| 命令引入宏包:
\begin{verbatim}
\usepackage{graphicx}
\end{verbatim}
引入 \verb|graphicx| 宏包后,就可以使用 \verb|\includegraphics| 命令插图了:
\begin{verbatim}
\includegraphics[width=3cm]{xiantu. pdf}
\end{verbatim}
然而插入的图形就是一个有内容的矩形盒子,在正文中和一个很大的字符没有多少区别。
因此会把插图和文件混在一起。
除了一些很小的标志图形,我们很少把插图直接夹在文字之中,而是使用单独的环境列出。
而且很大的图形如果固定位置,会给分页造成困难。
因此,通常都把图形放在一个可以变动相对位置的环境中,称为浮动体(float)。
在浮动体中还可以给图形加入说明性的标题,因此,在《杂谈勾股定理》中实际是使用下面的代码插图的:
\begin{verbatim}
\begin{figure}[ht]
  \centering
  \includegraphics[scale=0.6]{xiantu. pdf}
  \caption{宋赵爽在《周髀算经》注中作的弦图(仿制),该图给出了勾股定理的一个极具对称美的证明。}
  
  \label{fig: xiantu}
\end{figure}
\end{verbatim}
在上面的代码中,第 1 行和第 7 行使用了 \verb|figure| 环境,就是插图使用的浮动体环境。
\verb|figure| 环境有可选参数 \verb|[ht]|,表示浮动体可以出现在环境周围的文本所在处(here)和一页的顶部(top)。
\verb|figure| 环境内部相当于普通的段落(默认没有缩进);
第 2 行用声明 \verb|\centering| 表示后面的内容居中;
第 3 行插入图形;
第 4 行和第 5 行使用 \verb|\caption| 命令给插图加上自动编号和标题;
第 6 行的 \verb|\label| 命令则给图形定义一个标签,使用这个标签就可以在文章的其他地方引用 \verb|\caption| 产生的编号(编号引用我们会在后面讲到)。
采纳 \cmark。

下面再来看表格。
插图可以用其他软件做好插入,但表格一般都还是直接在 \LaTeX 里面完成的。
制作表格,需要确定的是表格的行、列对齐模式和表格线,这是由 \verb|tabular| 环境完成的:
\begin{tabular}{|rrr|}
\hline
直角边 $a$ &直角边 $b$ &斜边 $c$ \\
\hline
3&4&5\\
5&12&13\\
\hline
\end{tabular}


本打算从 1.2 节开始看,但是我急需存放演示代码,所以跳转到刘海洋前辈教材 \cite{刘海洋2013latex} 的 2.2.5 小节看看。


\subsection{2.1.5水平间距与盒子}%
\label{sub:2.1.5水平间距与盒子}

\paragraph{如何正确地使用水平间距}%
\label{par:如何正确地使用水平间距}

\paragraph{TeX的长度单位}%
\label{par:TeX的长度单位}

\TeX 的长度单位有如下几种:
pt,pc 相当于四号字大小,in inch 英寸。
\footnote{修改建议:改成列表}

最重要的长度单位是 em,全身长度,表示字号对应的长度,本义是大写字母 M 的长度。
我们可以这样检验它:
\begin{verbatim}
M \quad\quad\quad M
\end{verbatim}
M M M M M\\
MMMM~~~~M\\
M~\quad~\quad~\quad~M\\
MMMMM\\
M\quad\quad\quad{\!}M\\
M\quad\quad{\quad}M\\
M\quad\quad\quad~M\\
M\quad\quad\quad{\enskip}M\\
M\quad\quad\quad\!M\\
M\quad\quad\quad{\,}M\\
M\qquad\quad{M}\\

最简单的命令是 \verb|\mbox{<内容>}|,产生一个盒子,内容左右模式排列,不允许断行:
\mbox{aaaaaaaaaaaaaaaaaaaaaaaaaaaaaaaaaaaaaaaaaaaaaaaaaaaaaaaaaaaaaaaaaaaaaaaaaaaaaaaaaaaaaaaaaaaaaaa}。

\begin{verbatim}
\makebox[<宽度>][<位置>]{<内容>}
\end{verbatim}


\subsection{2.2.5抄录和代码环境}%
\label{sub:2.2.5抄录和代码环境}

\paragraph{先学会演示代码}%
\label{par:请先学会演示代码}

为什么我们需要 2.2.5 小节的知识,因为有时候我们必须经常性地使用特殊符号。
在 2.2.5 小节,刘海洋前辈介绍了抄录功能 verbatim。
在抄录环境里,我们能随意打印特殊符号。

\verb|\verb| 命令可用来表示行文中的抄录,它的语法是这样的:
\begin{verbatim}
\verb<符号><抄录内容><符号>
\end{verbatim}
其中,这里的符号可以是 \verb|"|、\verb|!|、\verb"|" 等。
\footnote{改进建议:我希望把语法演示部分放在框框里}

另一方面,大段抄录则可以使用 \verb|\verbatim| 环境。
它的语法是这样的:
\begin{verbatim}
begin{verbatim}
<抄录内容>
end{verbatim}
\end{verbatim}
\footnote{改进建议:暂时我没法展示它的语法,因为我没法在 verbatim 环境里嵌套 verbatim。}

至此,我们所要求的基本功能已经实现了。
我们对这个命令并不满意,理想中的代码演示环境:
\begin{enumerate}
  \item 
    能演示几乎所有代码,包括 \verb|verbatim| 环境
  \item 
    其次它最好能用框框框起来,左边是代码,右边是效果
  \item 
    它要能智能换行
\end{enumerate}

\paragraph{如何在抄录环境里显式显示空格}%
\label{par:如何在抄录环境里显式显示空格}

使用带星号的命令 \verb|\verb*| 使得输出的空格可见:\verb*| |。
同样可以使用带星号的 \verb|verbatim*| 环境输出可见空格:
\begin{verbatim*}
  hey, guy!
\end{verbatim*}

\paragraph{抄录环境和其他环境能否相互嵌套}%
\label{par:抄录环境和其他环境能否相互嵌套}

首先,\verb|verbatim| 环境并不能放在框框里。
因为抄录命令 \verb|\verb| 和 \verb|verbatim| 环境,一般不能作为其他命令的参数。
例如,\verb|\fbox{\verb!abc!}| 会报错
\begin{quote}
\zihao{-5}\kaishu
LaTeX Error: \verb illegal in argument.
\end{quote}
\verb|\fbox{<输入文本>}| 的效果是把输入文本放在框里,例如 \verb|\fbox{框里的文字}| 效果为 \fbox{框里的文字}。
\footnote{改进建议:把代码及其效果放在框里}

但是可以使用 fancyvrb 宏包提供的命令,使用 \verb|lrbox| 环境把它们保存在自定义盒子里面,参考 2.1.5 节。

\begin{verbatim}
\SaveVerb{myverb}|#$%^&|
\fbox{套中 \UseVerb{myverb}}。
\end{verbatim}
其效果为
\SaveVerb{myverb}|#$%^&|
\fbox{套中 \UseVerb{myverb}}。
fracyvrb 宏包功能强大,可参考文档 Zandt \cite{van2010fancyvrb}。

也可以利用 cprotect 宏包,它定义了 \verb|\cprotect| 等命令,可以方便地在其他命令参数中使用 \verb|\verb| 命令或者 \verb|verbatim| 环境。
\verb|\cprotect| 命令用在带参数的命令前,用于保护参数:\verb!\cprotect\fbox{套中 \verb|#$%^&|}!,其效果为 \cprotect\fbox{套中 \verb|#$%^&|}。
看起来,\verb|\cprotect| 更方便,但是它的限制更多,在一些命令 \verb|\parbox| 依然需要先保存再使用的方式。

\paragraph{实现verb的简写}%
\label{par:实现verb的简写}

shortvrb 宏包提供了 \verb|\verb| 的简写形式,可以使用 \verb|\MakeShortVerb<符号>| 来定义简写,\verb|\DeleteShortVerb<符号>| 来取消定义。

我还很想知道,怎么生成一个 box。

\paragraph{排版支持语法高亮的程序代码}%
\label{par:排版支持语法高亮的程序代码}

支持语法高亮功能的 listings 宏包。
%暂且不管,先用最基本的命令,行内代码使用 \verb|\verb| 命令,行间代码使用 \verb|\verbatim| 环境。

listing 宏包支持的基本功能是 \verb|lstlisting| 环境。
不过没有做任何设置的时候,\verb|lstlisting| 环境没有语法高亮的效果而且看起来比 \verb|verbatim| 环境还要难看。
怎么让 \verb|lstlisting| 环境更好看呢?
可以设置它的可选参数,也可以使用 \verb|\lstset{<设置>}| 进行全局设置:
\begin{verbatim}
\begin{lstlisting}[language=C]
/* hello.c */
#include <stdio.h>
main() {
    printf("Hello.\n");
}
\end{lstlisting}
\end{verbatim}
如此,C 语言的注释用斜体排出,关键字用粗体排出。
进一步,\verb|basicstyle| 选项可以设置整体格式,\verb|keywordstyle| 选项设置关键字,\verb|stringstyle| 设置字符串,\verb|identifierstyle| 设置标识符,\verb|commenstyle| 设置注释。\footnote{似乎更适合采用表格形式}。

为什么我看教材看得这么慢,这么痛苦?
因为我没有目标,也没有疑问。
明天早上起来学习,这样就有压力了。

接着看回刘海洋前辈教材的第1.2节。
写文章自然首先考虑文章的结构,具体地说,我们要写一篇科技论文,它的结构包括标题、摘要、目录,正文、参考文献。
有结构是好事,我希望我的文字有统一的结构,这样将会方便我后期进行统一处理。
然后它问我,我可以完成这样一个任务嘛?按照他的愿望,我需要自己的文章有主要的五大结构,其次还要能插入文字、公式、图片、表格等形式,最后设置页面大小,四合一。
难点有哪些呢?
首先是文章的结构,其次是文章的内容,最后是页面也就是视图方面的设置。

首先是文章的总体说明,文章采用的编码格式、文件名、文章的描述:
\begin{verbatim}
%-*- coding: UTF-8 -*-
% gougu.tex
% 勾股定理
\documentclass[UTF8]{ctexart}
\end{verbatim}
其次是文章的标题、作者、日期:
\begin{verbatim}
\title{}
\author{}
\date{\today}
\end{verbatim}
然后是文章的核心分块,\verb|doucment|,必须要生成标题和目录:
\begin{verbatim}
\begin{document}
\maketitle
\tableofcontents
\end{document}
\end{verbatim}
此外,打印文献的位置也是固定的。
\begin{verbatim}
\bibliographystyle{plain}
\bibliography{math}
\end{verbatim}
毫无疑问,这些都是初始化必要的,可以写入 \verb|snippets|,关键字 \verb|init| 扩展为这些内容。

摘要部分有专门的环境:
\begin{verbatim}
\begin{abstract}
这是一篇短文。
\end{abstract}
\end{verbatim}
其效果为:
\begin{abstract}
这是一篇短文。
\end{abstract}

输入文字的原则:
\begin{enumerate}
\item
  空行用来分段
\item
  单个换行相当于空格
\item
  汉字后的空格会被忽略
\item
  不要在句子开头使用汉字的全角空格
\item
  在汉字与其他符号之间加入空格,可令代码更加清晰
\end{enumerate}

学会使用脚注和引用。
脚注 \footnote{脚注} 由 \verb|\footnote{<脚注内容>}| 命令实现。
\emph{强调} 由 \verb|\emph{<强调内容>}| 命令实现。

\begin{verbatim}
\begin{quote}
引用部分由 \verb|quote| 环境实现。
但是引用的部分不会改变引用内容的字体,因此需要结合 {\zihao{-5}
\kaishu 改变字体的命令使用}。
\end{quote}
\end{verbatim}
其效果为:
\begin{quote}
引用部分由 \verb|quote| 环境实现。
但是引用的部分不会改变引用内容的字体,因此需要结合 {\zihao{-5}\kaishu 改变字体的命令使用}。
\end{quote}

输入数学公式是我的强项,但是怎么使用表格就很难。
插入图片由宏包 graphicx 提供:
\begin{verbatim}
\usepackage{graphicx}

\includegraphics[width=3cm]{xiantu.pdf}
\end{verbatim}
这种情况可能会导致插图和文字混在一起。
实际上,我们很少把插图直接夹在文字中,而是使用单独的环境列出。
还可以加入说明性的文字:
\begin{verbatim}
\begin{figure}[ht]
  \centering
  \includegraphics[scale=0.6]{xiantu.pdf}
  \caption{<自动编号和标题>}
  \label{<定义标签>}
\end{figure}
\end{verbatim}

插入表格:
\begin{verbatim}
\begin{tabular}{|rrr|}
 \hline
 <1> & <2> & <3> \\
 \hline
 <4> & <5> & <6> \\
 <7> & <8> & <9>
 \hline
\end{tabular}
\end{verbatim}
如果不想让表格浮动,可以这样
\begin{verbatim}
\begin{table}[H]
\begin{tabular}{|rrr|}
 \hline
 <1> & <2> & <3> \\
 \hline
 <4> & <5> & <6> \\
 <7> & <8> & <9>
 \hline
\end{tabular}
\end{table}
\end{verbatim}

gvim 似乎不能调用。

\bibliography{latex}

\end{document}
